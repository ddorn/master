% \newcommand{\lang}{french}  % Copy this in a french document
% \newcommand{\UseWhiteBackground}{1}  % Copy this to have a white document

\documentclass{scrarticle}
\usepackage[utf8]{inputenc}
\usepackage{standalone}
\usepackage{etoolbox}
\ifdef{\lang}{}{\newcommand{\lang}{english}}  % If no \lang command is defined, default to english
\usepackage[\lang]{babel}
\usepackage{amssymb,amsmath,amsthm,epsfig,eepic,epic,verbatim,moreverb,fancybox}
\usepackage{wasysym}
% \usepackage{bbold}  % NO! DON"T USE IT. It add mathbb for integers bet makes it ugly for letters.
\usepackage{dsfont}
% \usepackage{tocstyle}  % To change the table of contents (TOC) style, and that's what we do next
% \usetocstyle{nopagecolumn}  % One of classic, KOMAlike, nopagecolumn, allwithdot, noonewithdot, standard
\usepackage{tikz}
\usepackage{tikz-cd}
\usetikzlibrary{calc,math,decorations.markings}
\usetikzlibrary{shapes.geometric,patterns,automata}
\usepackage{rotating}  % To have large figures in lanscape with \begin{sidewaysfigure} ... \end{sidewaysfigure}
\usepackage{pdflscape}  % To change one page to landscape
\usepackage{multicol}  % For the \begin{multicol}{N} environement
\usepackage{wrapfig}   % For the \begin{wrapfig}{r|l}{4cm} environement that behave like html float
\usepackage{subcaption}  % To use the subfigure environement in figure environements
\usepackage{mathtools}  % For the \xSOMETHINGarrow that we can write on top and change length
\usepackage{xspace}  % To add a space in commands when the following is not pounctuation
\usepackage{todonotes}  % To add todos in the margin of a document with \todo. Also \listoftodos is nice !
\usepackage{graphicx}  % many operations like rotating text etc...
\usepackage{ifthen}
\usepackage{xifthen}
\usepackage{float}  % To have the [H] option for figure so that they don't float. Use only if really needed.
\usepackage{enumerate}  % To be able to change the numbering of enumerate environements with [i)] or [a.]...
\usepackage{enumitem}  % To manage the spacing of items and margins
\usepackage{forest}  % For automatic tikz layout
\usepackage[framemethod=tikz]{mdframed}  % Frame around theorems
\usepackage{thmtools}  % Fixes the issue with \autoref and shared counters for thm env. Probably lots of other things also...
\usepackage{listings}  

\usepackage{xcolor}
% creamy background
\ifdef{\UseWhiteBackground}{  % White, default
    \definecolor{myBGcolor}{HTML}{FFFFFF}
    \definecolor{myTextcolor}{HTML}{000000}
    \definecolor{linkColor}{HTML}{044F67}

    % \colorlet{linkColor}{orange!60!blue}
    \colorlet{myRed}{orange!25!myBGcolor}
    \colorlet{myGreen}{green!25!myBGcolor}
}{  % Cream
    \definecolor{myBGcolor}{HTML}{E6E0C6}
    \definecolor{myTextcolor}{HTML}{2F251C}
    \colorlet{linkColor}{orange!70!blue}

    \pagecolor{myBGcolor}
    \color{myTextcolor}

    \colorlet{myRed}{orange!15!myBGcolor}
    \colorlet{myGreen}{green!15!myBGcolor}
}


\usepackage{hyperref}  % Always include it last, as it overrides a lot of commands.
\hypersetup{
    bookmarks=true,
    % pdfpagemode=FullScreen,  % Only for presentations
    pdfauthor={Diego Dorn},
    pdftitle={ok},
    colorlinks=true,
    allcolors=linkColor
    % linkcolor=blue,
    % urlcolor={winered},
    % filecolor={winered},
    % citecolor={winered},
    % allcolors={orange},
    % linktoc=all,
}

% \usepackage{cleveref}  % Needs to be after hyperref


% Tool to translate commands
% The first argument is the english text, the second is in french
\newcommand{\ifenglish}[2]{\ifthenelse{\equal{\lang}{english}}{#1}{#2}}

% Frames
\mdfdefinestyle{theoremstyle}{%
roundcorner=0pt, % round corners, makes it friendlier
leftmargin=1pt, rightmargin=1pt, % this helps with box warnings
innerbottommargin=12pt,
hidealllines=true, % even friendlier
align=center, %
}

% Theorem environements
\theoremstyle{definition}   % Not in italics
\ifenglish{
    \newtheorem{theorem}{Theorem}[section]
    \newtheorem*{theorem*}{Theorem}
    \newtheorem{definition}[theorem]{Definition}
    \newtheorem{notation}[theorem]{Notation}
    \newtheorem{corollary}[theorem]{Corollary}
    \newtheorem{proposition}[theorem]{Proposition}
    \newtheorem*{proposition*}{Proposition}
    \newtheorem{lemma}[theorem]{Lemma}
    \newtheorem*{lemma*}{Lemma}
    \newtheorem*{remark}{Remark}
    \newtheorem*{claim}{Claim}
    \newtheorem*{example}{Example}
    \newtheorem*{examples}{Examples}
    \newtheorem*{exercise}{Exercise}
}{ 
    \newtheorem{theorem}{Théorème}[section]
    \newtheorem{definition}[theorem]{Définition}
    \newtheorem{notation}[theorem]{Notation}
    \newtheorem{corollary}[theorem]{Corrolaire}
    \newtheorem{proposition}[theorem]{Proposition}
    \newtheorem{lemma}[theorem]{Lemme}
    \newtheorem*{remark}{Remarque}
    \newtheorem*{claim}{Assertion}
    \newtheorem*{example}{Exemple}
    \newtheorem*{examples}{Exemples}
    \newtheorem*{exercise}{Exercise}
}


\surroundwithmdframed[style=theoremstyle,backgroundcolor=myRed]{theorem}
\surroundwithmdframed[style=theoremstyle,backgroundcolor=myGreen]{definition}

%%%%%%%%%%%%%%%%%%%
%   Parenthesis   %
%%%%%%%%%%%%%%%%%%%

\newcommand{\parenthesis}[1]{\left(#1\right)}
\newcommand{\bigparenthesis}[1]{\big(#1\big)}
\newcommand{\Bigparenthesis}[1]{\Big(#1\Big)}

\newcommand{\brackets}[1]{\left[#1\right]}

%%%%%%%%%%%%%%%%%%%
%    Operators    %
%%%%%%%%%%%%%%%%%%%

\DeclareMathOperator{\id}{id}
\DeclareMathOperator{\re}{Re}
\DeclareMathOperator{\Aut}{Aut}
\DeclareMathOperator{\Hom}{Hom}
\DeclareMathOperator{\Char}{char}
\DeclareMathOperator{\im}{Im}
\DeclareMathOperator{\dom}{dom}
\DeclareMathOperator{\cof}{cof}
\DeclareMathOperator{\Ob}{Ob}
\DeclareMathOperator{\Mor}{Mor}
\DeclareMathOperator{\rk}{rk}

\newcommand{\concat}{
  \mathord{
    \mathchoice
    {\raisebox{1ex}{\scalebox{.7}{$\frown$}}}
    {\raisebox{1ex}{\scalebox{.7}{$\frown$}}}
    {\raisebox{.7ex}{\scalebox{.5}{$\frown$}}}
    {\raisebox{.7ex}{\scalebox{.5}{$\frown$}}}
  }
}
\newcommand{\restr}[1]{\mathbin\upharpoonright_{#1}}
\newcommand{\compl}[1]{#1^\mathsf{c}}
\newcommand{\abs}[1]{\left| #1 \right|}
\newcommand{\bigabs}[1]{\big\lvert#1\big\rvert}
\newcommand{\parens}[1]{\left(#1\right)}
\newcommand{\norm}[1]{\left\| #1 \right\|}

%%%%%%%%%%%%%%%%%%%
%  Constructions  %
%%%%%%%%%%%%%%%%%%%

% Arrows
\newcommand{\xto}[1]{\xrightarrow{#1}}

% Structures 
\newcommand{\verteqsign}{\rotatebox{90}{$\,=$}}
\newcommand{\verteq}[2]{\underset{\scriptstyle\overset{\mkern4mu\verteqsign}{#2}}{#1}}
\newcommand{\vertimplies}[2]{\underset{\scriptstyle\overset{\uparrow}{#2}}{#1}}
\newcommand{\func}[5]{ 
    \begin{array}{rccl}
            #1 : & #2 & \longrightarrow & #3 \\
                 & #4 & \longmapsto     & #5
    \end{array} 
}
\newcommand{\ntimes}[2]{\underbrace{#1, \dots, #1}_{#2 \ttimes}}

% Set definitions
\newcommand{\set}[1]{\left\{ #1 \right\}}
\newcommand{\bigset}[1]{\big\{ #1 \big\}}
\newcommand{\setst}[2]{\left\{ #1 \, \middle| \, #2 \right\}}
\newcommand{\bigsetst}[2]{\big\{ #1 \, \big| \, #2 \big\}}

\newcommand{\funcset}[2]{{}^{#1}{#2}}
\newcommand{\angles}[1]{\left\langle #1 \right\rangle}
\newcommand{\brz}{B_r(z_0)}
\newcommand\quotient[2]{
    \mathchoice
        {% \displaystyle
            \text{\raise1ex\hbox{$#1$}\Big/\lower1ex\hbox{$#2$}}%
        }
        {% \textstyle
            #1\,/\,#2
        }
        {% \scriptstyle
            #1\,/\,#2
        }
        {% \scriptscriptstyle  
            #1\,/\,#2
        }
}
\newcommand{\interior}[1]{\mathring{#1}}

%%%%%%%%%%%%%%%
%   Letters   %
%%%%%%%%%%%%%%%


% Greek alphabe
\renewcommand{\phi}{\ensuremath\varphi\xspace}
\renewcommand{\epsilon}{\ensuremath\varepsilon\xspace}
\renewcommand{\a}{\ensuremath\alpha\xspace}
\renewcommand{\d}{\ensuremath{\partial}\xspace}
\newcommand{\D}{\ensuremath\Delta\xspace}
\newcommand{\e}{\ensuremath\epsilon\xspace}
\newcommand{\g}{\ensuremath\gamma\xspace}
\newcommand{\y}{\g}  % I mistake them every other time
\newcommand{\p}{\phi\xspace}
\newcommand{\s}{\sigma\xspace}
\newcommand{\w}{\ensuremath\omega\xspace}

% Mathbb alphabet
\newcommand{\C}{\ensuremath{\mathbb C}\xspace}
\newcommand{\F}{\ensuremath{\mathbb F}\xspace}
\newcommand{\IG}{\ensuremath{\mathbb G}\xspace}
\newcommand{\N}{\ensuremath{\mathbb N}\xspace}
\newcommand{\R}{\ensuremath{\mathbb R}\xspace}
\newcommand{\IP}{\ensuremath{\mathbb P}\xspace}
\newcommand{\Q}{\ensuremath{\mathbb Q}\xspace}
\newcommand{\Z}{\ensuremath{\mathbb Z}\xspace}

% Fancy fonts
\newcommand{\Aa}{\ensuremath{\mathcal A}\xspace}
\newcommand{\Bb}{\ensuremath{\mathcal B}\xspace}
\newcommand{\Cc}{\ensuremath{\mathcal C}\xspace}
\newcommand{\calG}{\mathcal{G}\xspace}
\newcommand{\calK}{\mathcal{K}\xspace}
\newcommand{\Dd}{\ensuremath{\mathcal D}\xspace}
\newcommand{\Ff}{\mathcal{F}}
\newcommand{\Pp}{\mathcal{P}}
\newcommand{\Mm}{\mathcal{M}}
\newcommand{\Nn}{\mathcal{N}}
\newcommand{\Tt}{\ensuremath{\mathcal T}\xspace}
\newcommand{\Ss}{\ensuremath{\mathcal S}\xspace}
\newcommand{\Uu}{\ensuremath{\mathcal U}\xspace}
\renewcommand{\O}[1]{\mathcal{O}\left(#1\right)}

% Others
\renewcommand{\emptyset}{\varnothing}


%%%%%%%%%%%%%%%%%%
%   New symbols  %
%%%%%%%%%%%%%%%%%%

\newcommand{\iddots}{{\cdot^{\cdot^\cdot}}}  % Three dots on the antidiagonal
\newcommand{\existsinf}{\exists^\infty}

%%%%%%%%%%%%%%%%%%
%   Shorthands   %
%%%%%%%%%%%%%%%%%%

% Math text
\newcommand{\open}{\text{ open }}
\newcommand{\st}{\text{ \ifenglish{s.t}{t.q}. }}
\newcommand{\andd}{\text{ \ifenglish{and}{et} }}
\newcommand{\andquad}{\quad\andd\quad}
\newcommand{\et}{\text{ et }}
\newcommand{\ttimes}{\text{ times}}
\newcommand{\tif}{\text{if }}
\newcommand{\otherwise}{\text{otherwise}}

% Symbols
\newcommand{\acts}{\circlearrowright}
\newcommand{\sub}{\subset}
\newcommand{\subeq}{\subseteq}
\renewcommand{\iff}{\ensuremath{\Longleftrightarrow}}
\newcommand{\iffquad}{\quad\iff\quad}
\newcommand{\del}{\partial}
\newcommand{\x}{\times}
\renewcommand{\emph}{\textbf}

%%%%%%%%%%%%%%%%%%%%
% Subject specific %
%%%%%%%%%%%%%%%%%%%%

% Maybe those should go inside their respesective files

% Graph theory
\DeclareMathOperator*{\ex}{ex}
\newcommand{\Turan}{\mathcal{T}}

% Analytic number theory
\newcommand{\sumdn}{\sum_{d|n}}
\newcommand{\Li}{\mathrm{Li}}

% Smooth manifolds
\newcommand{\chart}[1]{(\phi_{#1}, U_{#1})}
\newcommand{\der}[1]{\frac{\del}{\del #1}}

% Galois Theory
\newcommand{\clot}[1]{\overline{#1}}
\newcommand{\Gal}{\mathrm{Gal}}
% \DeclareMathOperator*{\Gal}{Gal}
\newcommand{\Kxx}{K[x_1, \dots, x_n]}

% Algebraic topology
\newcommand{\HCW}{H^{\mathrm{CW}}}

% Complexity
\newcommand{\coP}{\mathsf{coP}}
\newcommand{\Ptime}{\mathsf{P}}
\newcommand{\NP}{\mathsf{NP}}
\newcommand{\coNP}{\mathsf{coNP}}
\newcommand{\PSPACE}{\mathsf{PSPACE}}
\newcommand{\NPSPACE}{\mathsf{NPSPACE}}


% Descriptive Set Theory
\newcommand{\baire}{\ensuremath{\w^\w}\xspace}
\newcommand{\cantor}{\ensuremath{2^\w}\xspace}
\newcommand{\W}{\ensuremath{\w_1}\xspace}  % This is meant to be used only for the first uncountable ordinal
% Classes
\newcommand{\bS}[1]{\ensuremath{\mathbf{\Sigma}^0_{#1}}\xspace}
\newcommand{\bP}[1]{\ensuremath{\mathbf{\Pi}^0_{#1}}\xspace}
\newcommand{\bD}[1]{\ensuremath{\mathbf{\Delta}^0_{#1}}\xspace}
\newcommand{\borelsigma}[1]{\ensuremath{\mathbf{\Sigma}^0_{#1}}\xspace}
\newcommand{\borelpi}[1]{\ensuremath{\mathbf{\Pi}^0_{#1}}\xspace}
\newcommand{\boreldelta}[1]{\ensuremath{\mathbf{\Delta}^0_{#1}}\xspace}
\newcommand{\borelGamma}{\ensuremath{\mathbf{\Gamma}}\xspace}
\newcommand{\LambdaClass}[2]{\ensuremath{\mathbf{\Lambda}_{#1, #2}}\xspace}
\newcommand{\dualclass}[1]{\check{#1}}
\newcommand{\nsd}{non-self-dual\xspace}
\newcommand{\projectivesigma}[1]{\ensuremath{\mathbf{\Sigma}^1_{#1}}\xspace}
\newcommand{\projectivepi}[1]{\ensuremath{\mathbf{\Pi}^1_{#1}}\xspace}
\newcommand{\projectivedelta}[1]{\ensuremath{\mathbf{\Delta}^1_{#1}}\xspace}
% Sets of sequences
\newcommand{\seqlt}[2]{{#2}^{<#1}}
\newcommand{\finiteseq}[1]{\seqlt{\w}{#1}}
% \newcommand{\infseq}[1]{\funcset{\omega}{#1}}
% \newcommand{\allseq}[1]{\funcset{\leq\omega}{#1}}
\newcommand{\infseq}[1]{{#1}^\w}
\newcommand{\allseq}[1]{{#1}^{\leq\omega}}
\newcommand{\finitebaire}{\ensuremath{\finiteseq{\w}}\xspace}
\newcommand{\finitecantor}{\ensuremath{\finiteseq{2}}\xspace}
\newcommand{\length}[1]{\mathrm{lh}(#1)}
\newcommand{\emptyseq}{\langle\rangle}
% Operators
\renewcommand{\succ}[2][]{\ifthenelse{\isempty{#1}}{\mathrm{succ}(#2)}{\mathrm{succ}_{#1}(#2)}}
\DeclareMathOperator{\odd}{odd}
\DeclareMathOperator{\even}{even}
\DeclareMathOperator{\last}{last}
\newcommand{\Countf}[1]{\#_{#1}}
\newcommand{\difference}[1]{D_{#1}}
\newcommand{\suffixes}[2]{{#1}_{(#2)}}
\newcommand{\wadd}{\oplus}
\newcommand{\wCountableSup}{\mathrm{sup^\w}}
\newcommand{\wmult}{\mathop{\otimes}}
\newcommand{\bigeraser}{\ensuremath{\mathord{\Uparrow}}\xspace}
\newcommand{\interleave}{\mathbin{\S}}
\newcommand{\Init}[1]{\mathrm{Init}_{#1}}
% Players
\newcommand{\I}{\texttt{I}\xspace}
\newcommand{\II}{\texttt{II}\xspace}
\newcommand{\playerA}{Player \I}
\newcommand{\playerB}{Player \II}

% \renewcommand{\I}{\texttt{M}\xspace}
% \renewcommand{\II}{\texttt{B}\xspace}
% \renewcommand{\playerA}{Michael\xspace}
% \renewcommand{\playerB}{Bob\xspace}
% Wadge
\newcommand{\leW}{\le_\textsc{w}}
\newcommand{\geW}{\ge_\textsc{w}}
\newcommand{\gtW}{>_\textsc{w}}
\newcommand{\ltW}{<_\textsc{w}}
\newcommand{\equivW}{\equiv_\textsc{w}}
\newcommand{\Wdegree}[1]{[#1]_{\equivW}}
% Axioms / Theories
\newcommand{\ZF}{\textbf{ZF}\xspace}
\newcommand{\ZFC}{\textbf{ZFC}\xspace}
\newcommand{\AD}{\textbf{AD}\xspace}
\renewcommand{\AC}{\textbf{AC}\xspace}
\newcommand{\DC}{\textbf{DC}\xspace}
% Games
\newcommand{\GaleStewart}[1]{\ensuremath{G(#1)}\xspace}
\newcommand{\BanachMazur}[1]{\ensuremath{BM(#1)}\xspace}
\newcommand{\Wadge}[2]{\ensuremath{G(#1, #2)}\xspace}
% Automatons
\newcommand{\automaton}{automatic set\xspace}
\newcommand{\automata}{automatic sets\xspace}
\newcommand{\Automata}{Automatic sets\xspace}
\newcommand{\Language}{\mathcal{L}}
\DeclareMathOperator{\Inf}{Inf}
\newcommand{\AcceptanceCondition}{\mathrm{Acc}}
\newcommand{\automatonMinus}{\scalebox{0.5}{\begin{tikzpicture}[automata, baseline=-2mm]
    \node[minus] {};
\end{tikzpicture}}\xspace}
\newcommand{\automatonPlus}{\scalebox{0.5}{\begin{tikzpicture}[automata, baseline=-2mm]
    \node[plus] {};
\end{tikzpicture}}\xspace}
\newcommand{\automatonState}[1]{\scalebox{0.5}{\begin{tikzpicture}[automata, baseline=-2mm, minimum size=20]
    \node[state] {\Large \(#1\)};
\end{tikzpicture}}\xspace}
\newcommand{\sumAutomaton}[2]{\begin{tikzpicture}[automata]
    \path node[initial, state] (root) {$#2$} 
        +(2, 1) node[state] (A) {$#1$}
        +(2, -1) node[state] (Ac) {$\compl{#1}$};
    \draw (root) edge (A) edge (Ac);
\end{tikzpicture}}


%%%%%%%%%%%%%%%%%
%     Tikz      %
%%%%%%%%%%%%%%%%%

\tikzcdset{row sep/normal=1cm, column sep/normal=1cm}  % So tikz-cd diagram are more squared

% Draws irregular circles.
\newcommand{\irregularcircle}[2]{  % radius, irregularity
  let \n1 = {(#1)+rand*(#2)} in
  +(0:\n1)
  \foreach \a in {10,20,...,350}{
    let \n1 = {(#1)+rand*(#2)} in
    -- +(\a:\n1)
  } -- cycle
}

\tikzset{
triangle/.style={ % To create triangles in trees
  draw=myRed!90!black,
  text=black,
  fill=myRed,
  shape border uses incircle,
  isosceles triangle,
  shape border rotate=90,
}}

% Used to make arrows on paths. Use like this: \draw[marr=\Singlearrow]
\tikzset{marr/.style={
  decoration={
    markings,
    mark=at position 0.5 with {#1}
    },
  postaction={decorate}
}}
\def\Singlearrow{{\arrow[scale=1.5,xshift={0.5pt+2.25\pgflinewidth}]{>}}}
\def\Doublearrow{{\arrow[scale=1.5,xshift=1.35pt +2.47\pgflinewidth]{>>}}}
\def\Triplearrow{{\arrow[scale=1.5,xshift=1.75pt +2.47\pgflinewidth]{>>>}}}
\newcommand{\CWarrow}[3]{ [->] ($1/3*(#1) + 1/3*(#2) + 1/3*(#3)$) ++(220:4pt) arc (220:-40:4pt)}
\newcommand{\CCWarrow}[3]{ [->] ($1/3*(#1) + 1/3*(#2) + 1/3*(#3)$) ++(-40:4pt) arc (-40:220:4pt)}

% For automatons
\tikzset{
    initial text={},
    do path picture/.style={%
        path picture={%
        \pgfpointdiff{\pgfpointanchor{path picture bounding box}{south west}}%
            {\pgfpointanchor{path picture bounding box}{north east}}%
        \pgfgetlastxy\x\y%
        \tikzset{x=\x/2,y=\y/2}%
        #1
        }
    },
    automata/.style={
        node distance=60,  % I don't know what the unit is. I would like 2.5cm, but that can scale
        minimum size=20,
        every edge/.style={
            draw,
            ->,
            auto,
        }
    },
    plus/.style={
        draw,
        circle,
        do path picture={    
            \draw [line cap=round, very thick] (-1/2,0) -- (1/2,0) (0,-1/2) -- (0,1/2);
        }
    },
    minus/.style={
        draw,
        circle,
        do path picture={    
            \draw [line cap=round, very thick] (-1/2,0) -- (1/2,0);
        }
    },
    % Removes extra space on initial nodes
    % From: https://tex.stackexchange.com/questions/111554/how-to-avoid-superfluous-space-in-the-initial-state-of-a-tikz-automaton#comment245757_111558
    every initial by arrow/.append style={anchor/.append style={shape=coordinate}},
}

% To put at the bottom right corner of a square that commutes.
\newcommand{\commutes}{\arrow[ul, phantom, "\scalebox{1.5}{$\circlearrowleft$}"]}

% To link two lines in a long exact sequence.
% First argument is where the line should connect (ex: dll)
% Second argument is the label on the line.
\newcommand{\connecting}[2]{
    \arrow[#1,phantom, ""{coordinate, name=Z}]
    \arrow[#1, "#2", rounded corners,
        to path={
            -- ([xshift=2ex]\tikztostart.east)
            |- (Z)[near start]\tikztonodes
            -| ([xshift=-2ex]\tikztotarget.west)
            -- (\tikztotarget)
        }]
}
\newcommand{\dotsto}{\mathllap{\cdots\longrightarrow\ }}  % ... → indicate the start of a long sequence without taking space
\newcommand{\todots}{\mathrlap{\ \longrightarrow\cdots}}  % → ... indicate the end of a long sequence without taking space 

\newenvironment{longsequence}{
    \begin{center}
        \begin{tikzcd}[row sep=0.5cm]
}{
        \end{tikzcd}
    \end{center}
}


\newcommand{\quickfig}[2]{
    \begin{wrapfigure}{r}{40mm}
        \begin{center}
            #2
        \end{center}
        \caption{#1}
    \end{wrapfigure}
}



\title{Analysis on Groups}
\author{Diego Dorn \\ Lectures notes from Nicolas Monod}
\date{Automn 2021}

\newcommand{\Isom}{\mathrm{Isom}}
\newcommand{\Acts}{\curvearrowright}

\begin{document}
    \maketitle

    % \tableofcontents

    \section{Means and amenable actions}

    \paragraph{Setting} $G$ is a group, 
    $X$ is a $G$-set, that is, a set with an action of $G$, 
    \[
        \func{a}{G \times X}{X}{(g, x)}{gx}
    \]
    That verify \begin{itemize}
        \item $\forall x, ex = x$
        \item $\forall x \in X, \forall g, h \in G, g(hx) = (gh)(x)$
    \end{itemize}

    \begin{example}
        $X = \R^3$ and $G = \Isom(\R^3)$
    \end{example}

    \begin{definition}
        A \emph{mean} on a set $X$ is a map 
        $\mu: \Pp(X) \to [0, 1]$
        such that \begin{itemize}
            \item $\mu(X) = 1$
            \item $\forall A, B \subseteq X$ disjoint, 
                $\mu(A \cup B) = \mu(A) + \mu(B)$.
        \end{itemize}
    \end{definition}

    \begin{exercise}
        Prove that 
        \begin{enumerate}
            \item $\mu(\emptyset) = 0$
            \item $\forall A, B, \mu(A \cup B) = \mu(A) + \mu(B) - \mu(A \cap B)$.
            \item $\forall n, \forall A_1, \dots A_n \subseteq X$ disjoint,
                $\mu(\bigcup_{i = 1}^n A_i) = \sum_{i = 1}^n \mu(A_i)$.
            \item If $A \subset B$ then $\mu(A) \leq \mu(B)$.
        \end{enumerate}
    \end{exercise}

    \begin{definition}
        The action $G \Acts X$ is called \emph{amenable}
        if there exists a $G$-invariant mean on $X$
    \end{definition}

    Explocitly, \[
        \forall g \in G~
        \forall A \subseteq X,
        \mu(A) = \mu(gA). 
    \]
    This prevents and "Banach-Tarski" paradox in $X$ with respect to $G$.

    \subsection{More abstract means}

    Let $\Mm(X)$ be the set of means on $X$.
    \begin{example}
        Let $x \in X$, then $\d_x$ is the \emph{Dirac mass of $x$} defined by
        \[
            \d_x(A) = \begin{cases} 
                0 & \tif x \notin A \\
                1 & \tif x \in A.
            \end{cases} 
        \]
        Then $\d_x \in \Mm(X)$.
    \end{example}

    \begin{lemma}
        If $\mu, \mu' \in \Mm(X)$ then
        $\frac{1}{2}(\mu + \mu') \in \mu(X)$.
        More generally, for $\lambda \in [0, 1]$, 
        \[
            \lambda\mu + (1 - \lambda)\mu' \in \Mm(X).
        \]
    \end{lemma}

    \begin{corollary}
        Any convex combination of means is a mean.
        \[
            \sum_{i = 1}^n \lambda_i \mu_i
        \]
        With $\lambda_i \leq 0$ and $\sum \lambda_i = 1$.
    \end{corollary}

    \begin{remark}
        $\Mm(X)$ is a subset of the vector space $\R^{\Pp(X)}$
        of all functions $\Pp(X) \to \R$.
        We endow $\R^{\Pp(X)}$ with the product topology, 
        which is the topology of pointwise convergence.
        Thus a sequence $(\mu_n)$ of means converges to $\mu$
        iff $\forall A \subseteq X, \mu_n(A) \to \mu(A)$.
    \end{remark}

    \begin{proposition}
        $\Mm(X)$ is compact.
    \end{proposition}

    \begin{proof}
        $\Mm(X) \subset [0, 1]^{\Pp(X)}$ which by Tychonov is compact.
        Thus we only need to show that $\Mm(X)$ is closed.
        We an write $\Mm(X)$ as:
        \begin{align*}
            \Mm(X) =&
            \setst{
                \mu \in [0, 1]^{\Pp(X)}
            }{
                \mu(X) = 1
            }
            ~\cap \\&
            \bigcap_{A, B \text{ disjoint}}
            \setst{
                \mu \in [0, 1]^{\Pp(X)}
            }{
                \mu(A \cup B) = \mu(A) + \mu(B)
            }
        \end{align*}
        Since all those sets are defined by closed conditions, 
        they are closed and thus $\Mm(X)$ is closed.
    \end{proof}

    \subsection{Amenable groups}
    \paragraph{Functoriality for $\Mm(X)$}

    Let $f: X \to Y$ be a map. 
    Then $f$ induces a map $f_* : \Mm(X) \to \Mm(Y)$.
    For $A \subset Y$:
     \[
        (f_* \mu)(A) := \mu(f^{-1}(A))
    \]

    Given a group action $G \Acts X$ we obtain an action $G \Acts \Mm(X)$.  

    \begin{definition}
        An \emph{invariant mean} on $X$ is then a $G$-fixed point 
        in $M(X)$.
        We write $\Mm(X)^G = \setst{\mu \in \Mm(X)}{ \forall g \in G, g\mu = \mu}$ 
    \end{definition}

    In this sense, $G \Acts X$ is amenable if and only if
    $\Mm(X)^G \neq \emptyset$.

    \begin{definition}
        The group $G$ is \emph{amenable} 
        if the action $G \Acts G$ given by multiplication
        is an amenable action.
    \end{definition}

    \begin{example}
        Let $X \neq \emptyset$ be a finite $G$-set.
        Then $G \Acts X$ is amenable.

        Thus finite groups are amenable.
    \end{example}
    \begin{proof}
        Let $\mu = \frac{1}{|X|} \sum_{x \in X} \delta_x$ and $g \in G$
        \[
            g\mu = \frac{1}{|X|} \sum_{x \in X} g\delta_x
            = \frac{1}{|X|} \sum_{x \in X} \delta_{gx}
            = \mu
        \]
    \end{proof}

    In fact, the $G$-sets with a finite orbite are the only
    easy groups that are amenable.
    In particular, finite groups are the only easy example of amenable groups.
    Without the axiom of choice they are the only groups that
    are provably amenable.

    If $X$ is infinite, then there must be "exotic" (ie. non dirac masses)
    in $\Mm(X)$ since it is compact.
    When taking a sequence of Dirac masses, it must have a accumulation point.

    \begin{theorem}
        \Z is amenable.
    \end{theorem}

    \begin{proof}
        For $n \in \N$ let 
        $\mu_n = \frac{1}{n} \sum_{j = 1}^n \delta_j \in \Mm(\Z)$.
        Let $\mu \in \Mm(\Z)$ be an accumumlation point 
        of $(\mu_n)$, which exists since $\Mm(\Z)$ is compact.

        Warning: $(\mu_n)$ has no converging subsequence! 

        By definition of an accumulation point, for each neighbourhood $U$
        of $\mu$, $\forall N \exists n > N$ such that $\mu_n \in U$.
        That is, for all $A \in \Z$, $\mu(A)$ is an accumulation point of
        $(\mu_n(A))_{n \in \N}$.

        To prove that $\mu$ is \Z-invariant, we write $\Z = \angles{u}$ 
        multiplicatively. To show that $g\mu = \mu, \forall g \in \Z$,
        it suffices to show that $u\mu = \mu$.
        It is also enough to show that $u\mu_n - \mu_n \to 0$ in $\R^{\Pp(\Z)}$
        as they ... 

        Here $u \delta_j = \delta_{j + 1}$, so $
            u\mu_n =
             u \frac{1}{n} \sum_{j = 1}^n \delta_j 
             =
             \frac{1}{n} \sum_{j = 2}^{n+1} \delta_j 
            $
        So $u\mu_n - \mu_n = \frac{1}{n}\parenthesis{\delta_{n+1} - \delta_1}$
        since the sum is telescopic.
        Thus, for all $A \subset \Z$,
        $(u\mu_n - \mu_n)(A)
         = \frac{1}{n}\parenthesis{\delta_{n+1}(A) - \delta_1(A)}
         \to 0$
    \end{proof}

    \subsection{Towards an example of non-amenable groups}

    \begin{definition}
        Let $S$ be any set. The \emph{free group on S, $F_S$} is 
        constructed as follows
        \begin{itemize}
            \item take a disjoint copy $S^{-1}$ of $S$,
                ie. a disjoint set with a bijection $S \to S^{-1}: s \mapsto s^{-1}$
            \item Let \[
                \parenthesis{S \sqcup S^{-1}}^* = \bigsqcup_{n = 0}^\infty
                    \parenthesis{S \sqcup S^{-1}}^n
                \]
                We endow this set with concatenation to make it a monoid.
                The neutral element is $\e := \emptyset$.
            \item Consider the equivalence relation on 
                $\parenthesis{S \sqcup S^{-1}}^*$ generated by $(a, a^{-1}) \sim e$
                and $(a^{-1}, a) \sim e$
                for all $a \in S$ and stable under concatenation.
                The quotient is a group, $F_s$.
            \item Alternatively, we call a word in $\parenthesis{S \sqcup S^{-1}}^*$
                \emph{reduced} if it contains no instance of $(a, a^{-1})$ or 
                $(a^{-1}, a)$, for $a \in S$. The group operation is however 
                to concatenate then reduce.
        \end{itemize}
    \end{definition}

    \begin{tikzcd}
        S \rar{"map"} \ar{d}{\subseteq}
            & G
        \\
        F_S \ar[sloped]{ur}{\exists! \text{ homo}}
    \end{tikzcd}

    \paragraph{Facts about $F_s$}
    \begin{itemize}
        \item $F_s$ depends only on $\abs{S}$ up to isomorphism,
            so we write $F_n$ if $\abs{S} = n \in \N$.
        \item $F_0 = \set{e}$
        \item $F_1 \simeq \Z$
        \item For $n \geq 2$, $F_n$ is not abelan.
    \end{itemize}

    \begin{theorem}
        \begin{itemize}
            \item Every subgroup of a free group is free.
            \item For all $n \in \N$, $F_n < F_2$.
        \end{itemize}
    \end{theorem}

    \begin{theorem}
        $F_2$ is not amenable.
    \end{theorem}
    
    \begin{proof}
        Suppose for a contradiction that we have $\mu \in \Pp(F_2)^{F_2}$.
        Write $F_2 = {e} \sqcup A_+ \sqcup A_- \sqcup B_+ \sqcup B_-$, 
        where $F_2 = F_{\set{a, b}}$ and $A_+$ is the set of reduced words 
        starting with $a$ (\dots).

        Note that $a^{-1} A_+ = F_2 \setminus A_-$.

        Now \begin{align*}
            \mu(A_+) 
            &= \mu(a^{-1}A_+)
            \\&= \mu(\sqcup A_+ \sqcup B_+ \sqcup B_-)
            \\&= \mu(\sqcup A_+) + 
                \underbrace{
                    \mu(\sqcup B_+) + \mu(\sqcup B_-) 
                }_{=0}
        \end{align*}

        So the measure for each part is 0, and $F_2 = 0 \neq 1$
    \end{proof}

    \begin{lemma}
        Let $X, Y$ be $G$-sets. Let $f: X \to Y$ be a $G$-map (i.e. $f(gx) = gf(x)$)
        If $G \Acts X$ is amenable, then so is $G \Acts Y$.
    \end{lemma}

    \begin{proof}
        Functoriality.
    \end{proof}

    \begin{corollary}
        If $G$ is amenable, then any $G$-action is amenable
        (on a $X \neq 0$).
    \end{corollary}

    \begin{proof}
        Use corollary with the $G$-map $\func{f}{G}{X}{g}{f(g) = g \cdot x_0}$
         for some $x_0 \in X$.
    \end{proof}

    \begin{definition}
        An action $G \Acts X$ is \emph{free} 
        XXX ... 
    \end{definition}

    \begin{proposition}
        If $G$ admits an amenable free action, then $G$ is amenable.
    \end{proposition}
    \begin{proof}
        Let $G \Acts X$ be a free amenable action.
        Let $R \subseteq X$ be a set of representatives of the $G$-orbits.
        Define $\func{f}{G\times R}{X}{(g, r)}{gr}$, then $f$ is surjective.

        We claim that the freeness implies $f$ injective.
        Indeed if $g'r' = gr$, then $g^{-1}g'r' = r$
        so $r'$ and $r$ are in the same orbit and $r = r'$.
    \end{proof}

    \begin{corollary}
        Every subgroup of an amenable group is amenable.
    \end{corollary}

    \begin{corollary}
        If a group contains an non-amenable subgroup, then
        it is not amenable.
    \end{corollary}

    Temporary definition:
    \[
        \Mm'(X) := \setst{
            m \in l^\infty(X)^*
        }{
            \begin{array}{rc}
                1) & m \geq 0 \\
                2) & m(\mathds{1}_X) = 1 \\
                3) & \norm{m} \leq 1
            \end{array}
        }
    \]

    There is a natural and linear map \[ 
        \func{a}{\Mm'(X)}{\Mm(X)}
            {m}{\mu_m}
    \]
    defined by $\mu_m(A) = m(\mathbb{1}_A)$.

    \begin{theorem}
        Under this map, we have an isomorphism:
        \[
            \Mm'(X) \cong \Mm(x)
        \]
    \end{theorem}

    \begin{proof}
        Let \[
            \Ff = \setst{ f: X \to \R}{ f(X) \text{ finite}}
        \]
        Note that \[
            \Ff = \setst{ \sum_{i = 1}^n \lambda_i \mathbb{1}_{A_i} }
                { A_i \text{ disjoint}}
        \]

    \end{proof}
    
    \section{Fixed point theorems}

    \subsection{Barycenter}
    \begin{definition}
        A \emph{convex compact set} is a subset $K \subset V$
        of a topological vector space $V$ which is convex and compact.
    \end{definition}

    \begin{definition}
        Consider a mean $\mu \in \Mm(K)$ on a convex compact set $K \subset V$.

        A point $x \in V$ is a \emph{barycenter} for $\mu$ 
        if $\mu(\lambda|_K) = \lambda(x)$ for all $\lambda \in V^*$.
    \end{definition}

    \begin{theorem}
        There exists a barycenter in $K$.
    \end{theorem}

    \begin{theorem}
        Let $G$ be a group.
        $G$ is amenable if and only if 
        for all convex compact $G$-space $K \neq \emptyset$
        there exists a fixed point $K^G \neq \emptyset$.
    \end{theorem}

    \section{The class of amenable groups}

    So far:
    \begin{itemize}
        \item Finite groupes and \Z are amenable.
        \item Amenability passes to subgroups (proved with free actions).
        \item Amenability passes to quotient groups.
    \end{itemize}

    \begin{definition}
        $G$ is an \emph{extension} of $N$ by $Q$
        if there exists $N \lhd G$ and $\quotient{G}{N} = Q$.
    \end{definition}

    Notation:
    \begin{tikzcd}
        1 \rar 
            & N \rar
                & G \rar
                    & Q \rar
                        & 1
    \end{tikzcd}

    Examples:
    \begin{itemize}
        \item $G = N \times Q$.
        \item $G$ the Heisenberg group
            \[
                \setst{
                    \begin{pmatrix}
                        1 & x & z \\
                        0 & 1 & y \\
                        0 & 0 & 1
                    \end{pmatrix}
                }{
                    x, y, z \in \Z
                }
            \]
            with \[
                N = \setst{
                    \begin{pmatrix}
                        1 & 0 & z \\
                        0 & 1 & 0 \\
                        0 & 0 & 1
                    \end{pmatrix}
                }{
                    x, y, z \in \Z
                }
            \]
            Note that \[
                \begin{pmatrix}
                    1 & x & z \\
                    0 & 1 & y \\
                    0 & 0 & 1
                \end{pmatrix}
                \begin{pmatrix}
                    1 & x' & z' \\
                    0 & 1 & y' \\
                    0 & 0 & 1
                \end{pmatrix}
                = 
                \begin{pmatrix}
                    1 & x+x' & z+z' + xy' \\
                    0 & 1 & y+y' \\
                    0 & 0 & 1
                \end{pmatrix}
            \]
            And $\quotient{G}{N} = \Z^2$.
            Here \begin{itemize}
                \item $G$ is not commutative;
                \item $\Z^2 \not< G$;
                \item $G \neq \Z^3$.
            \end{itemize}
    \end{itemize}

    \begin{proposition}
        Extensions preserve amenability.
    \end{proposition}

    \begin{corollary}~
        \begin{enumerate}
            \item $\Z^n$ is amenable for all $n$.
            \item The Heisenberg group is amenable.
        \end{enumerate}
    \end{corollary}

    \begin{corollary}
        Every finitely generated abelian grou is amenable.
    \end{corollary}

    \begin{definition}
        A net $\Ff$ of subgroups $H < G$ of a group $G$
        is \emph{directed} if 
        $\forall H, H' \in \Ff, \exists H'' \geq H, H'$

        We say yhat $G$ is a \emph{directed union} if
        $G = \bigcup_{H \in \Ff} H$.
    \end{definition}

    \begin{proposition}
        Directed unions preserve amenability.
    \end{proposition}

    \begin{corollary}
        Every abelian group is amenable.
    \end{corollary}

    \begin{definition}
        \emph{Soluble} groups are all the groups obtained by finitely 
        many extensions starting with abelian groups.
    \end{definition}

    \begin{corollary}
        Every soluble group is amenable.
    \end{corollary}

    \begin{definition}
        Let $(G_i)_{i \in I}$ be any family of group.
        The \emph{direct sum} or \emph{restricted direct product}
        is 
        \[
            \bigoplus_{i \in I} G_i
            \bigsetst{
                (g_i)_{i \in I}
            }{
                \exists F \subseteq I \text{ finite},
                g_i = e \forall F i \notin F
            }.
        \]
    \end{definition}
    \newpage
    .
\end{document}
