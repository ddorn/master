% \newcommand{\lang}{french}  % Copy this in a french document
% \newcommand{\UseWhiteBackground}{1}  % Copy this to have a white document

\documentclass{scrarticle}
\usepackage[utf8]{inputenc}
\usepackage{standalone}
\usepackage{etoolbox}
\ifdef{\lang}{}{\newcommand{\lang}{english}}  % If no \lang command is defined, default to english
\usepackage[\lang]{babel}
\usepackage{amssymb,amsmath,amsthm,epsfig,eepic,epic,verbatim,moreverb,fancybox}
\usepackage{wasysym}
% \usepackage{bbold}  % NO! DON"T USE IT. It add mathbb for integers bet makes it ugly for letters.
\usepackage{dsfont}
% \usepackage{tocstyle}  % To change the table of contents (TOC) style, and that's what we do next
% \usetocstyle{nopagecolumn}  % One of classic, KOMAlike, nopagecolumn, allwithdot, noonewithdot, standard
\usepackage{tikz}
\usepackage{tikz-cd}
\usetikzlibrary{calc,math,decorations.markings}
\usetikzlibrary{shapes.geometric,patterns,automata}
\usepackage{rotating}  % To have large figures in lanscape with \begin{sidewaysfigure} ... \end{sidewaysfigure}
\usepackage{pdflscape}  % To change one page to landscape
\usepackage{multicol}  % For the \begin{multicol}{N} environement
\usepackage{wrapfig}   % For the \begin{wrapfig}{r|l}{4cm} environement that behave like html float
\usepackage{subcaption}  % To use the subfigure environement in figure environements
\usepackage{mathtools}  % For the \xSOMETHINGarrow that we can write on top and change length
\usepackage{xspace}  % To add a space in commands when the following is not pounctuation
\usepackage{todonotes}  % To add todos in the margin of a document with \todo. Also \listoftodos is nice !
\usepackage{graphicx}  % many operations like rotating text etc...
\usepackage{ifthen}
\usepackage{xifthen}
\usepackage{float}  % To have the [H] option for figure so that they don't float. Use only if really needed.
\usepackage{enumerate}  % To be able to change the numbering of enumerate environements with [i)] or [a.]...
\usepackage{enumitem}  % To manage the spacing of items and margins
\usepackage{forest}  % For automatic tikz layout
\usepackage[framemethod=tikz]{mdframed}  % Frame around theorems
\usepackage{thmtools}  % Fixes the issue with \autoref and shared counters for thm env. Probably lots of other things also...
\usepackage{listings}  

\usepackage{xcolor}
% creamy background
\ifdef{\UseWhiteBackground}{  % White, default
    \definecolor{myBGcolor}{HTML}{FFFFFF}
    \definecolor{myTextcolor}{HTML}{000000}
    \definecolor{linkColor}{HTML}{044F67}

    % \colorlet{linkColor}{orange!60!blue}
    \colorlet{myRed}{orange!25!myBGcolor}
    \colorlet{myGreen}{green!25!myBGcolor}
}{  % Cream
    \definecolor{myBGcolor}{HTML}{E6E0C6}
    \definecolor{myTextcolor}{HTML}{2F251C}
    \colorlet{linkColor}{orange!70!blue}

    \pagecolor{myBGcolor}
    \color{myTextcolor}

    \colorlet{myRed}{orange!15!myBGcolor}
    \colorlet{myGreen}{green!15!myBGcolor}
}


\usepackage{hyperref}  % Always include it last, as it overrides a lot of commands.
\hypersetup{
    bookmarks=true,
    % pdfpagemode=FullScreen,  % Only for presentations
    pdfauthor={Diego Dorn},
    pdftitle={ok},
    colorlinks=true,
    allcolors=linkColor
    % linkcolor=blue,
    % urlcolor={winered},
    % filecolor={winered},
    % citecolor={winered},
    % allcolors={orange},
    % linktoc=all,
}

% \usepackage{cleveref}  % Needs to be after hyperref


% Tool to translate commands
% The first argument is the english text, the second is in french
\newcommand{\ifenglish}[2]{\ifthenelse{\equal{\lang}{english}}{#1}{#2}}

% Frames
\mdfdefinestyle{theoremstyle}{%
roundcorner=0pt, % round corners, makes it friendlier
leftmargin=1pt, rightmargin=1pt, % this helps with box warnings
innerbottommargin=12pt,
hidealllines=true, % even friendlier
align=center, %
}

% Theorem environements
\theoremstyle{definition}   % Not in italics
\ifenglish{
    \newtheorem{theorem}{Theorem}[section]
    \newtheorem*{theorem*}{Theorem}
    \newtheorem{definition}[theorem]{Definition}
    \newtheorem{notation}[theorem]{Notation}
    \newtheorem{corollary}[theorem]{Corollary}
    \newtheorem{proposition}[theorem]{Proposition}
    \newtheorem*{proposition*}{Proposition}
    \newtheorem{lemma}[theorem]{Lemma}
    \newtheorem*{lemma*}{Lemma}
    \newtheorem*{remark}{Remark}
    \newtheorem*{claim}{Claim}
    \newtheorem*{example}{Example}
    \newtheorem*{examples}{Examples}
    \newtheorem*{exercise}{Exercise}
}{ 
    \newtheorem{theorem}{Théorème}[section]
    \newtheorem{definition}[theorem]{Définition}
    \newtheorem{notation}[theorem]{Notation}
    \newtheorem{corollary}[theorem]{Corrolaire}
    \newtheorem{proposition}[theorem]{Proposition}
    \newtheorem{lemma}[theorem]{Lemme}
    \newtheorem*{remark}{Remarque}
    \newtheorem*{claim}{Assertion}
    \newtheorem*{example}{Exemple}
    \newtheorem*{examples}{Exemples}
    \newtheorem*{exercise}{Exercise}
}


\surroundwithmdframed[style=theoremstyle,backgroundcolor=myRed]{theorem}
\surroundwithmdframed[style=theoremstyle,backgroundcolor=myGreen]{definition}

%%%%%%%%%%%%%%%%%%%
%   Parenthesis   %
%%%%%%%%%%%%%%%%%%%

\newcommand{\parenthesis}[1]{\left(#1\right)}
\newcommand{\bigparenthesis}[1]{\big(#1\big)}
\newcommand{\Bigparenthesis}[1]{\Big(#1\Big)}

\newcommand{\brackets}[1]{\left[#1\right]}

%%%%%%%%%%%%%%%%%%%
%    Operators    %
%%%%%%%%%%%%%%%%%%%

\DeclareMathOperator{\id}{id}
\DeclareMathOperator{\re}{Re}
\DeclareMathOperator{\Aut}{Aut}
\DeclareMathOperator{\Hom}{Hom}
\DeclareMathOperator{\Char}{char}
\DeclareMathOperator{\im}{Im}
\DeclareMathOperator{\dom}{dom}
\DeclareMathOperator{\cof}{cof}
\DeclareMathOperator{\Ob}{Ob}
\DeclareMathOperator{\Mor}{Mor}
\DeclareMathOperator{\rk}{rk}

\newcommand{\concat}{
  \mathord{
    \mathchoice
    {\raisebox{1ex}{\scalebox{.7}{$\frown$}}}
    {\raisebox{1ex}{\scalebox{.7}{$\frown$}}}
    {\raisebox{.7ex}{\scalebox{.5}{$\frown$}}}
    {\raisebox{.7ex}{\scalebox{.5}{$\frown$}}}
  }
}
\newcommand{\restr}[1]{\mathbin\upharpoonright_{#1}}
\newcommand{\compl}[1]{#1^\mathsf{c}}
\newcommand{\abs}[1]{\left| #1 \right|}
\newcommand{\bigabs}[1]{\big\lvert#1\big\rvert}
\newcommand{\parens}[1]{\left(#1\right)}
\newcommand{\norm}[1]{\left\| #1 \right\|}

%%%%%%%%%%%%%%%%%%%
%  Constructions  %
%%%%%%%%%%%%%%%%%%%

% Arrows
\newcommand{\xto}[1]{\xrightarrow{#1}}

% Structures 
\newcommand{\verteqsign}{\rotatebox{90}{$\,=$}}
\newcommand{\verteq}[2]{\underset{\scriptstyle\overset{\mkern4mu\verteqsign}{#2}}{#1}}
\newcommand{\vertimplies}[2]{\underset{\scriptstyle\overset{\uparrow}{#2}}{#1}}
\newcommand{\func}[5]{ 
    \begin{array}{rccl}
            #1 : & #2 & \longrightarrow & #3 \\
                 & #4 & \longmapsto     & #5
    \end{array} 
}
\newcommand{\ntimes}[2]{\underbrace{#1, \dots, #1}_{#2 \ttimes}}

% Set definitions
\newcommand{\set}[1]{\left\{ #1 \right\}}
\newcommand{\bigset}[1]{\big\{ #1 \big\}}
\newcommand{\setst}[2]{\left\{ #1 \, \middle| \, #2 \right\}}
\newcommand{\bigsetst}[2]{\big\{ #1 \, \big| \, #2 \big\}}

\newcommand{\funcset}[2]{{}^{#1}{#2}}
\newcommand{\angles}[1]{\left\langle #1 \right\rangle}
\newcommand{\brz}{B_r(z_0)}
\newcommand\quotient[2]{
    \mathchoice
        {% \displaystyle
            \text{\raise1ex\hbox{$#1$}\Big/\lower1ex\hbox{$#2$}}%
        }
        {% \textstyle
            #1\,/\,#2
        }
        {% \scriptstyle
            #1\,/\,#2
        }
        {% \scriptscriptstyle  
            #1\,/\,#2
        }
}
\newcommand{\interior}[1]{\mathring{#1}}

%%%%%%%%%%%%%%%
%   Letters   %
%%%%%%%%%%%%%%%


% Greek alphabe
\renewcommand{\phi}{\ensuremath\varphi\xspace}
\renewcommand{\epsilon}{\ensuremath\varepsilon\xspace}
\renewcommand{\a}{\ensuremath\alpha\xspace}
\renewcommand{\d}{\ensuremath{\partial}\xspace}
\newcommand{\D}{\ensuremath\Delta\xspace}
\newcommand{\e}{\ensuremath\epsilon\xspace}
\newcommand{\g}{\ensuremath\gamma\xspace}
\newcommand{\y}{\g}  % I mistake them every other time
\newcommand{\p}{\phi\xspace}
\newcommand{\s}{\sigma\xspace}
\newcommand{\w}{\ensuremath\omega\xspace}

% Mathbb alphabet
\newcommand{\C}{\ensuremath{\mathbb C}\xspace}
\newcommand{\F}{\ensuremath{\mathbb F}\xspace}
\newcommand{\IG}{\ensuremath{\mathbb G}\xspace}
\newcommand{\N}{\ensuremath{\mathbb N}\xspace}
\newcommand{\R}{\ensuremath{\mathbb R}\xspace}
\newcommand{\IP}{\ensuremath{\mathbb P}\xspace}
\newcommand{\Q}{\ensuremath{\mathbb Q}\xspace}
\newcommand{\Z}{\ensuremath{\mathbb Z}\xspace}

% Fancy fonts
\newcommand{\Aa}{\ensuremath{\mathcal A}\xspace}
\newcommand{\Bb}{\ensuremath{\mathcal B}\xspace}
\newcommand{\Cc}{\ensuremath{\mathcal C}\xspace}
\newcommand{\calG}{\mathcal{G}\xspace}
\newcommand{\calK}{\mathcal{K}\xspace}
\newcommand{\Dd}{\ensuremath{\mathcal D}\xspace}
\newcommand{\Ff}{\mathcal{F}}
\newcommand{\Pp}{\mathcal{P}}
\newcommand{\Mm}{\mathcal{M}}
\newcommand{\Nn}{\mathcal{N}}
\newcommand{\Tt}{\ensuremath{\mathcal T}\xspace}
\newcommand{\Ss}{\ensuremath{\mathcal S}\xspace}
\newcommand{\Uu}{\ensuremath{\mathcal U}\xspace}
\renewcommand{\O}[1]{\mathcal{O}\left(#1\right)}

% Others
\renewcommand{\emptyset}{\varnothing}


%%%%%%%%%%%%%%%%%%
%   New symbols  %
%%%%%%%%%%%%%%%%%%

\newcommand{\iddots}{{\cdot^{\cdot^\cdot}}}  % Three dots on the antidiagonal
\newcommand{\existsinf}{\exists^\infty}

%%%%%%%%%%%%%%%%%%
%   Shorthands   %
%%%%%%%%%%%%%%%%%%

% Math text
\newcommand{\open}{\text{ open }}
\newcommand{\st}{\text{ \ifenglish{s.t}{t.q}. }}
\newcommand{\andd}{\text{ \ifenglish{and}{et} }}
\newcommand{\andquad}{\quad\andd\quad}
\newcommand{\et}{\text{ et }}
\newcommand{\ttimes}{\text{ times}}
\newcommand{\tif}{\text{if }}
\newcommand{\otherwise}{\text{otherwise}}

% Symbols
\newcommand{\acts}{\circlearrowright}
\newcommand{\sub}{\subset}
\newcommand{\subeq}{\subseteq}
\renewcommand{\iff}{\ensuremath{\Longleftrightarrow}}
\newcommand{\iffquad}{\quad\iff\quad}
\newcommand{\del}{\partial}
\newcommand{\x}{\times}
\renewcommand{\emph}{\textbf}

%%%%%%%%%%%%%%%%%%%%
% Subject specific %
%%%%%%%%%%%%%%%%%%%%

% Maybe those should go inside their respesective files

% Graph theory
\DeclareMathOperator*{\ex}{ex}
\newcommand{\Turan}{\mathcal{T}}

% Analytic number theory
\newcommand{\sumdn}{\sum_{d|n}}
\newcommand{\Li}{\mathrm{Li}}

% Smooth manifolds
\newcommand{\chart}[1]{(\phi_{#1}, U_{#1})}
\newcommand{\der}[1]{\frac{\del}{\del #1}}

% Galois Theory
\newcommand{\clot}[1]{\overline{#1}}
\newcommand{\Gal}{\mathrm{Gal}}
% \DeclareMathOperator*{\Gal}{Gal}
\newcommand{\Kxx}{K[x_1, \dots, x_n]}

% Algebraic topology
\newcommand{\HCW}{H^{\mathrm{CW}}}

% Complexity
\newcommand{\coP}{\mathsf{coP}}
\newcommand{\Ptime}{\mathsf{P}}
\newcommand{\NP}{\mathsf{NP}}
\newcommand{\coNP}{\mathsf{coNP}}
\newcommand{\PSPACE}{\mathsf{PSPACE}}
\newcommand{\NPSPACE}{\mathsf{NPSPACE}}


% Descriptive Set Theory
\newcommand{\baire}{\ensuremath{\w^\w}\xspace}
\newcommand{\cantor}{\ensuremath{2^\w}\xspace}
\newcommand{\W}{\ensuremath{\w_1}\xspace}  % This is meant to be used only for the first uncountable ordinal
% Classes
\newcommand{\bS}[1]{\ensuremath{\mathbf{\Sigma}^0_{#1}}\xspace}
\newcommand{\bP}[1]{\ensuremath{\mathbf{\Pi}^0_{#1}}\xspace}
\newcommand{\bD}[1]{\ensuremath{\mathbf{\Delta}^0_{#1}}\xspace}
\newcommand{\borelsigma}[1]{\ensuremath{\mathbf{\Sigma}^0_{#1}}\xspace}
\newcommand{\borelpi}[1]{\ensuremath{\mathbf{\Pi}^0_{#1}}\xspace}
\newcommand{\boreldelta}[1]{\ensuremath{\mathbf{\Delta}^0_{#1}}\xspace}
\newcommand{\borelGamma}{\ensuremath{\mathbf{\Gamma}}\xspace}
\newcommand{\LambdaClass}[2]{\ensuremath{\mathbf{\Lambda}_{#1, #2}}\xspace}
\newcommand{\dualclass}[1]{\check{#1}}
\newcommand{\nsd}{non-self-dual\xspace}
\newcommand{\projectivesigma}[1]{\ensuremath{\mathbf{\Sigma}^1_{#1}}\xspace}
\newcommand{\projectivepi}[1]{\ensuremath{\mathbf{\Pi}^1_{#1}}\xspace}
\newcommand{\projectivedelta}[1]{\ensuremath{\mathbf{\Delta}^1_{#1}}\xspace}
% Sets of sequences
\newcommand{\seqlt}[2]{{#2}^{<#1}}
\newcommand{\finiteseq}[1]{\seqlt{\w}{#1}}
% \newcommand{\infseq}[1]{\funcset{\omega}{#1}}
% \newcommand{\allseq}[1]{\funcset{\leq\omega}{#1}}
\newcommand{\infseq}[1]{{#1}^\w}
\newcommand{\allseq}[1]{{#1}^{\leq\omega}}
\newcommand{\finitebaire}{\ensuremath{\finiteseq{\w}}\xspace}
\newcommand{\finitecantor}{\ensuremath{\finiteseq{2}}\xspace}
\newcommand{\length}[1]{\mathrm{lh}(#1)}
\newcommand{\emptyseq}{\langle\rangle}
% Operators
\renewcommand{\succ}[2][]{\ifthenelse{\isempty{#1}}{\mathrm{succ}(#2)}{\mathrm{succ}_{#1}(#2)}}
\DeclareMathOperator{\odd}{odd}
\DeclareMathOperator{\even}{even}
\DeclareMathOperator{\last}{last}
\newcommand{\Countf}[1]{\#_{#1}}
\newcommand{\difference}[1]{D_{#1}}
\newcommand{\suffixes}[2]{{#1}_{(#2)}}
\newcommand{\wadd}{\oplus}
\newcommand{\wCountableSup}{\mathrm{sup^\w}}
\newcommand{\wmult}{\mathop{\otimes}}
\newcommand{\bigeraser}{\ensuremath{\mathord{\Uparrow}}\xspace}
\newcommand{\interleave}{\mathbin{\S}}
\newcommand{\Init}[1]{\mathrm{Init}_{#1}}
% Players
\newcommand{\I}{\texttt{I}\xspace}
\newcommand{\II}{\texttt{II}\xspace}
\newcommand{\playerA}{Player \I}
\newcommand{\playerB}{Player \II}

% \renewcommand{\I}{\texttt{M}\xspace}
% \renewcommand{\II}{\texttt{B}\xspace}
% \renewcommand{\playerA}{Michael\xspace}
% \renewcommand{\playerB}{Bob\xspace}
% Wadge
\newcommand{\leW}{\le_\textsc{w}}
\newcommand{\geW}{\ge_\textsc{w}}
\newcommand{\gtW}{>_\textsc{w}}
\newcommand{\ltW}{<_\textsc{w}}
\newcommand{\equivW}{\equiv_\textsc{w}}
\newcommand{\Wdegree}[1]{[#1]_{\equivW}}
% Axioms / Theories
\newcommand{\ZF}{\textbf{ZF}\xspace}
\newcommand{\ZFC}{\textbf{ZFC}\xspace}
\newcommand{\AD}{\textbf{AD}\xspace}
\renewcommand{\AC}{\textbf{AC}\xspace}
\newcommand{\DC}{\textbf{DC}\xspace}
% Games
\newcommand{\GaleStewart}[1]{\ensuremath{G(#1)}\xspace}
\newcommand{\BanachMazur}[1]{\ensuremath{BM(#1)}\xspace}
\newcommand{\Wadge}[2]{\ensuremath{G(#1, #2)}\xspace}
% Automatons
\newcommand{\automaton}{automatic set\xspace}
\newcommand{\automata}{automatic sets\xspace}
\newcommand{\Automata}{Automatic sets\xspace}
\newcommand{\Language}{\mathcal{L}}
\DeclareMathOperator{\Inf}{Inf}
\newcommand{\AcceptanceCondition}{\mathrm{Acc}}
\newcommand{\automatonMinus}{\scalebox{0.5}{\begin{tikzpicture}[automata, baseline=-2mm]
    \node[minus] {};
\end{tikzpicture}}\xspace}
\newcommand{\automatonPlus}{\scalebox{0.5}{\begin{tikzpicture}[automata, baseline=-2mm]
    \node[plus] {};
\end{tikzpicture}}\xspace}
\newcommand{\automatonState}[1]{\scalebox{0.5}{\begin{tikzpicture}[automata, baseline=-2mm, minimum size=20]
    \node[state] {\Large \(#1\)};
\end{tikzpicture}}\xspace}
\newcommand{\sumAutomaton}[2]{\begin{tikzpicture}[automata]
    \path node[initial, state] (root) {$#2$} 
        +(2, 1) node[state] (A) {$#1$}
        +(2, -1) node[state] (Ac) {$\compl{#1}$};
    \draw (root) edge (A) edge (Ac);
\end{tikzpicture}}


%%%%%%%%%%%%%%%%%
%     Tikz      %
%%%%%%%%%%%%%%%%%

\tikzcdset{row sep/normal=1cm, column sep/normal=1cm}  % So tikz-cd diagram are more squared

% Draws irregular circles.
\newcommand{\irregularcircle}[2]{  % radius, irregularity
  let \n1 = {(#1)+rand*(#2)} in
  +(0:\n1)
  \foreach \a in {10,20,...,350}{
    let \n1 = {(#1)+rand*(#2)} in
    -- +(\a:\n1)
  } -- cycle
}

\tikzset{
triangle/.style={ % To create triangles in trees
  draw=myRed!90!black,
  text=black,
  fill=myRed,
  shape border uses incircle,
  isosceles triangle,
  shape border rotate=90,
}}

% Used to make arrows on paths. Use like this: \draw[marr=\Singlearrow]
\tikzset{marr/.style={
  decoration={
    markings,
    mark=at position 0.5 with {#1}
    },
  postaction={decorate}
}}
\def\Singlearrow{{\arrow[scale=1.5,xshift={0.5pt+2.25\pgflinewidth}]{>}}}
\def\Doublearrow{{\arrow[scale=1.5,xshift=1.35pt +2.47\pgflinewidth]{>>}}}
\def\Triplearrow{{\arrow[scale=1.5,xshift=1.75pt +2.47\pgflinewidth]{>>>}}}
\newcommand{\CWarrow}[3]{ [->] ($1/3*(#1) + 1/3*(#2) + 1/3*(#3)$) ++(220:4pt) arc (220:-40:4pt)}
\newcommand{\CCWarrow}[3]{ [->] ($1/3*(#1) + 1/3*(#2) + 1/3*(#3)$) ++(-40:4pt) arc (-40:220:4pt)}

% For automatons
\tikzset{
    initial text={},
    do path picture/.style={%
        path picture={%
        \pgfpointdiff{\pgfpointanchor{path picture bounding box}{south west}}%
            {\pgfpointanchor{path picture bounding box}{north east}}%
        \pgfgetlastxy\x\y%
        \tikzset{x=\x/2,y=\y/2}%
        #1
        }
    },
    automata/.style={
        node distance=60,  % I don't know what the unit is. I would like 2.5cm, but that can scale
        minimum size=20,
        every edge/.style={
            draw,
            ->,
            auto,
        }
    },
    plus/.style={
        draw,
        circle,
        do path picture={    
            \draw [line cap=round, very thick] (-1/2,0) -- (1/2,0) (0,-1/2) -- (0,1/2);
        }
    },
    minus/.style={
        draw,
        circle,
        do path picture={    
            \draw [line cap=round, very thick] (-1/2,0) -- (1/2,0);
        }
    },
    % Removes extra space on initial nodes
    % From: https://tex.stackexchange.com/questions/111554/how-to-avoid-superfluous-space-in-the-initial-state-of-a-tikz-automaton#comment245757_111558
    every initial by arrow/.append style={anchor/.append style={shape=coordinate}},
}

% To put at the bottom right corner of a square that commutes.
\newcommand{\commutes}{\arrow[ul, phantom, "\scalebox{1.5}{$\circlearrowleft$}"]}

% To link two lines in a long exact sequence.
% First argument is where the line should connect (ex: dll)
% Second argument is the label on the line.
\newcommand{\connecting}[2]{
    \arrow[#1,phantom, ""{coordinate, name=Z}]
    \arrow[#1, "#2", rounded corners,
        to path={
            -- ([xshift=2ex]\tikztostart.east)
            |- (Z)[near start]\tikztonodes
            -| ([xshift=-2ex]\tikztotarget.west)
            -- (\tikztotarget)
        }]
}
\newcommand{\dotsto}{\mathllap{\cdots\longrightarrow\ }}  % ... → indicate the start of a long sequence without taking space
\newcommand{\todots}{\mathrlap{\ \longrightarrow\cdots}}  % → ... indicate the end of a long sequence without taking space 

\newenvironment{longsequence}{
    \begin{center}
        \begin{tikzcd}[row sep=0.5cm]
}{
        \end{tikzcd}
    \end{center}
}


\newcommand{\quickfig}[2]{
    \begin{wrapfigure}{r}{40mm}
        \begin{center}
            #2
        \end{center}
        \caption{#1}
    \end{wrapfigure}
}


\title{Homework 3}
\author{Diego Dorn}
% \date{Automn 2021}

\newcommand{\Eq}{\textsc{Eq}\xspace}
\newcommand{\Iso}{\textsc{Iso}\xspace}
\newcommand{\REF}{\textsc{Ref}\xspace}
\newcommand{\SI}{\textsc{SI}\xspace}
\newcommand{\Rcc}[1]{\mathrm{R}^{cc}_{1/3}\parenthesis{#1}}
\newcommand{\Dcc}[1]{\mathrm{D}^{cc}\parenthesis{#1}}


\begin{document}
    \maketitle

    \section*{Exercise 1}

    \begin{claim}
        $f$ has sensitivity $s(f) \leq \O{n}$.
    \end{claim}
    \begin{proof}
        We first show that $s(f) \leq \O{n}$.
        Let $x \in \set{0, 1}^{n^2}$ be an input of the problem.
        We want to show that $s(f, x) \leq \O{n}$. 

        If $f(x) = 0$,
        we show that each row has at most 3 sensitive bits, which yields the result.
        Let $w$ be the number of ways to make the row paired.
        \begin{itemize}
            \item if a row has 0, 2, or 4+ bits set, changing a bit produces 
                string of odd hamming weight and thus not paired, so $w = 0$.
            \item if a row has one bit set, flipping one of its neighbours
                are the only way to change the output, so $w \leq 2$.
            \item if a row has three bits set, flipping any of those three bits
                are the only way to get to exactly two bits set, and thus there are
                at most three ways to pair the row, so $w \leq 3$.
        \end{itemize}

        Now if $f(x) = 1$, and there is more than one row which is paired,
        then there is no way to make them all of them unpaired by flipping only one 
        bit, so $s(f, x) = 0$.

        Finally, if $f(x) = 1$, and there is exactly one paired row, the only
        ways to make the row unpaired is by flipping one of the bits of the row,
        so $s(f, x) = n$.
        
        In each case, $s(f, x) \leq 3n$ so $s(f) \leq \O{n}$.
    \end{proof}

    \begin{claim}
        $f$ has block sensitivity $bs(f) \geq \Omega(n^2)$.
    \end{claim}
    \begin{proof}
        It suffices to exibit an input $x$ of $f$ which has block-sensitivity
        $bs(f, x) = \Omega(n^2)$.
        We take $x = 0^{n^2}$, and for each $i = 1, \dots, n^2/2$
        the subsets $S_i = \set{2i-1, 2i}$.
        Clearly, the $S_i$ are disjoints and for each $i$, $f(x^{S_i}) = 1 \neq f(x) = 0$.
        Thus, $bs(f, x) \geq n^2/2$ and therefore $bs(f) \geq n^2/2 \geq \Omega(n^2)$.
    \end{proof}

    \section*{Exercice 2}
    \paragraph{a)}
    We know that $\Rcc{\Eq_{n^2}} = \O{1}$, so
    we reduce $\Iso_n$ to it, which will prove that $\Rcc{\Iso_n} = \O{1}$.
    To that extent, both Alice and Bob compute a cannonical form of their
    graph, for instance the graph in the equivalence class of their graph
    with the lexicographically least adjency matrix. This adjency matrix
    can be represented as a $n^2$ bit string, 
    and the two strings will be equal if and only if the graphs were isomorphic.

    Since the best randomized protocol for equality satisfies 
    $\Rcc{\Eq_{n^2}} = \O{1}$, the same holds for $\Iso_n$.

    \paragraph{b)}
    We know that 
    $\Dcc{\Eq_n} = \Omega(n)$
    and thus
    $\Dcc{\Eq_{n^2}} = \Omega(n^2)$.
    We reduce $\Eq_{n^2}$ to $\Iso_{7n}$
    which shows that $\Dcc{\Iso_n} = \Omega(n^2)$.

    From a $n^2$ bit string $x$, we construct the graph $G_x$
    on $7n$ vertices with labels $k_1, \dots, k_{3n}$, $s_1, \dots, s_{2n}$ and $a_1, \dots, a_{2n}$
    by describing which edges belong to the graph:
    \begin{itemize}
        \item Each pair $k_i$, $k_j$ is connected, 
            making a copy of $K_{3n}$, the complete graph on $3n$ vertices.
        \item For $i = 1, \dots, 2n$, there is an edge between 
            $a_i$ and $k_1, \dots, k_i$, 
            making $a_i$ have exactly $i$ neighbours in the copy of $K_{3n}$.
        \item For $i = 1, \dots, 2n$, there is an edge between 
            $a_i$ and $s_i$.
        \item Let $b: [n^2] \to \setst{\set{a_i, a_j}}{i \neq j \in [2n]}$ be any injection from $[n^2]$ to the set of edges 
            between $a_i$'s. This injection exists, since the size of the set of edges 
            is $\frac{2n(2n - 1)}{2} \geq n^2$.
            Finally, for each $i \in [n^2]$ such that $x_i = 1$,
            there is the edge $b(i)$.
    \end{itemize}

    For instance for $n = 3$, we get this graph:

    \begin{center}
    \begin{tikzpicture}
        \node[regular polygon, draw] (K3n) {$K_{3n}$};
        \foreach \x in {1, ..., 6} {
            \node[draw, circle] (a\x) at ($(K3n) + (\x - 3.5, -2)$) {$a_{\x}$};
            \foreach \i in {1, ..., \x} \draw (a\x) edge[out=90, in=-90+\x*20-4*20+8*\i] (K3n) ;
            \node[draw, circle] (s\x) at ($(a\x) + (0, -1.5)$) {$s_{\x}$} edge (a\x);
            \foreach \i in {1, ..., \x} 
                \draw (s\x) edge[dashed,out=-90,in=-90] (s\i);
        };
    \end{tikzpicture}
    \end{center}
    Where the edges between the $s_i$ are set or not depending on
    the string $x$. As a sanity check, note that here there are 
    15 edges that can be set which leaves enough place to encode
    the $n^2 = 3^2 = 9$ bits in the string $x$.

    We now show that the function that maps $(x, y) \mapsto (G_x, G_y)$
    is indeed a reduction from $\Eq_{n^2}$ to $\Iso_{7n}$.
    First, if $x = y$, $G_x = G_y$ by construction and the two
    graphs are isomorphic.
    
    For the reverse direction, assume that $G_x \simeq G_y$,
    and the the isomorphism is realised by $f: G_x \to G_y$.
    The two graphs are on the same set of vertices, $k_i$, $a_i$, $s_i$ 
    so $f$ corresponds to a permutation of those vertices.
    If we show that $f(s_i) = s_i$
    we will be done, as for all $i \in [n^2]$, 
    \[
        x_i = 1 \iff b(i) \in E(G_x) \iff f(b(i)) \in E(G_y)
        \iff b(i) \in G(y) \iff y_i = 1
    \],
    where the second last equivalence is because $f$ fixes each $s_i$.

    To show that $f(s_i) = s_i$ for all $i \in [2n]$, we use
    the fact that the set of vertices of $G_x$ that verify a given property
    must be mapped by $f$ to the set which verify the same
    property in $G_y$.
    We then proceed in three steps:
    \begin{itemize}
        \item Let $K = \setst{k_i}{i \in [3n]}$ be the set of all $k_i$.
            Then $f(K) = K$, since $K$ is the unique $3n$-clique in both $G_x$
            and $G_y$.
        \item For $i \in [2n]$, $f(a_i) = a_i$, indeed, for both graphs,
            $a_i$ is the only vertex adjacent to $K$ with degree $i + 1$.
        \item For $i \in [2n]$, $f(s_i) = s_i$, indeed $s_i$ is the only
            neighbour of $a_i$ not in $K$, and since both $a_i$ and $K$
            are fixed, we must have $f(s_i) = s_i$.
    \end{itemize}

    Therefore, $x = y \iff G_x \simeq G_y$ and $\Eq_{n^2}$ reduces to $\Iso_{7n}$,
    so $\Iso_{7n} \geq \Omega(n^2)$ and $\Iso_n \geq \Omega(n^2)$.\qed
    

    \section*{Exercice 3}
    We know that $\Dcc{\SI_n} = \Omega(n)$, where $\SI$ is
    the set-intersection problem,
    and therefore $\Dcc{\compl\SI_n} = \Omega(n)$,
    where $\compl\SI$ is the complement of $\SI$,
    i.e. set-non-intersection.
    We reduce $\compl\SI_n$ to $\REF_{n+1}$ to prove the lower bound.

    Let $a \in \set{0, 1}^n$ and $b \in \set{0, 1}^n$ be Alice and Bob's input
    for $\compl\SI_n$ respectively.
    We define $A = \setst{i \in [n]}{a_i = 1} \cup \set{n + 1}$
    and similarly $B = \setst{i \in [n]}{b_i = 0} \cup \set{n + 1}$.
    Notice that:
    \begin{align*}
        (a, b) \in \compl\SI_n 
        &\iff a \cap b = \emptyset 
        \\& \iff \forall i \in [n],~ a_i = 1 \implies b_i = 0
        \\& \iff A \subset B
    \end{align*}

    Our reduction to $\REF_{n+1}$ is then 
    \[
        \mathcal P = \set{A} \cup 
        \bigcup_{\substack{i \in [n] \\ a_i = 0}}
            \set{\set{i}}
    \]
    \[
        \mathcal Q = \set{B} \cup 
        \bigcup_{\substack{i \in [n] \\ b_i = 1}}
            \set{\set{i}}
    \]
    which are the partitions containing $A$ (resp. $B$) and the rest as singletons.

    It remains to show that $\mathcal P \sqsubseteq \mathcal Q \iff a \cap b = \emptyset$, but
    \begin{align*}
        \mathcal P \sqsubseteq \mathcal Q 
        & \iff \forall P \in \mathcal P, \exists Q \in \mathcal Q,~ P \subset Q
        \\& \iff \exists Q \in \mathcal Q,~ A \subset Q
        \\& \iff A \subset B
        \\& \iff (a, b) \in \compl\SI_n
    \end{align*}
    Where the second line comes from the fact that the condition is always met
    for singletons, and the only non-singleton element of $\mathcal P$
    is $A$. The third line comes from the fact that $n + 1 \in A$
    and the only set that contains $n + 1$ in $\mathcal Q$ is $B$, 
    so if $A$ is a subset of some part of $\mathcal Q$, it must be $B$.

    Thus, solving $\compl\SI_n$ is simpler than solving $\REF_{n+1}$ and 
    can be done only in $\Omega(n)$, so the same is true for $\REF_{n+1}$

\end{document}
