% \newcommand{\lang}{french}  % Copy this in a french document
% \newcommand{\UseWhiteBackground}{1}  % Copy this to have a white document

\documentclass{scrarticle}
\usepackage[utf8]{inputenc}
\usepackage{standalone}
\usepackage{etoolbox}
\ifdef{\lang}{}{\newcommand{\lang}{english}}  % If no \lang command is defined, default to english
\usepackage[\lang]{babel}
\usepackage{amssymb,amsmath,amsthm,epsfig,eepic,epic,verbatim,moreverb,fancybox}
\usepackage{wasysym}
% \usepackage{bbold}  % NO! DON"T USE IT. It add mathbb for integers bet makes it ugly for letters.
\usepackage{dsfont}
% \usepackage{tocstyle}  % To change the table of contents (TOC) style, and that's what we do next
% \usetocstyle{nopagecolumn}  % One of classic, KOMAlike, nopagecolumn, allwithdot, noonewithdot, standard
\usepackage{tikz}
\usepackage{tikz-cd}
\usetikzlibrary{calc,math,decorations.markings}
\usetikzlibrary{shapes.geometric,patterns,automata}
\usepackage{rotating}  % To have large figures in lanscape with \begin{sidewaysfigure} ... \end{sidewaysfigure}
\usepackage{pdflscape}  % To change one page to landscape
\usepackage{multicol}  % For the \begin{multicol}{N} environement
\usepackage{wrapfig}   % For the \begin{wrapfig}{r|l}{4cm} environement that behave like html float
\usepackage{subcaption}  % To use the subfigure environement in figure environements
\usepackage{mathtools}  % For the \xSOMETHINGarrow that we can write on top and change length
\usepackage{xspace}  % To add a space in commands when the following is not pounctuation
\usepackage{todonotes}  % To add todos in the margin of a document with \todo. Also \listoftodos is nice !
\usepackage{graphicx}  % many operations like rotating text etc...
\usepackage{ifthen}
\usepackage{xifthen}
\usepackage{float}  % To have the [H] option for figure so that they don't float. Use only if really needed.
\usepackage{enumerate}  % To be able to change the numbering of enumerate environements with [i)] or [a.]...
\usepackage{enumitem}  % To manage the spacing of items and margins
\usepackage{forest}  % For automatic tikz layout
\usepackage[framemethod=tikz]{mdframed}  % Frame around theorems
\usepackage{thmtools}  % Fixes the issue with \autoref and shared counters for thm env. Probably lots of other things also...
\usepackage{listings}  

\usepackage{xcolor}
% creamy background
\ifdef{\UseWhiteBackground}{  % White, default
    \definecolor{myBGcolor}{HTML}{FFFFFF}
    \definecolor{myTextcolor}{HTML}{000000}
    \definecolor{linkColor}{HTML}{044F67}

    % \colorlet{linkColor}{orange!60!blue}
    \colorlet{myRed}{orange!25!myBGcolor}
    \colorlet{myGreen}{green!25!myBGcolor}
}{  % Cream
    \definecolor{myBGcolor}{HTML}{E6E0C6}
    \definecolor{myTextcolor}{HTML}{2F251C}
    \colorlet{linkColor}{orange!70!blue}

    \pagecolor{myBGcolor}
    \color{myTextcolor}

    \colorlet{myRed}{orange!15!myBGcolor}
    \colorlet{myGreen}{green!15!myBGcolor}
}


\usepackage{hyperref}  % Always include it last, as it overrides a lot of commands.
\hypersetup{
    bookmarks=true,
    % pdfpagemode=FullScreen,  % Only for presentations
    pdfauthor={Diego Dorn},
    pdftitle={ok},
    colorlinks=true,
    allcolors=linkColor
    % linkcolor=blue,
    % urlcolor={winered},
    % filecolor={winered},
    % citecolor={winered},
    % allcolors={orange},
    % linktoc=all,
}

% \usepackage{cleveref}  % Needs to be after hyperref


% Tool to translate commands
% The first argument is the english text, the second is in french
\newcommand{\ifenglish}[2]{\ifthenelse{\equal{\lang}{english}}{#1}{#2}}

% Frames
\mdfdefinestyle{theoremstyle}{%
roundcorner=0pt, % round corners, makes it friendlier
leftmargin=1pt, rightmargin=1pt, % this helps with box warnings
innerbottommargin=12pt,
hidealllines=true, % even friendlier
align=center, %
}

% Theorem environements
\theoremstyle{definition}   % Not in italics
\ifenglish{
    \newtheorem{theorem}{Theorem}[section]
    \newtheorem*{theorem*}{Theorem}
    \newtheorem{definition}[theorem]{Definition}
    \newtheorem{notation}[theorem]{Notation}
    \newtheorem{corollary}[theorem]{Corollary}
    \newtheorem{proposition}[theorem]{Proposition}
    \newtheorem*{proposition*}{Proposition}
    \newtheorem{lemma}[theorem]{Lemma}
    \newtheorem*{lemma*}{Lemma}
    \newtheorem*{remark}{Remark}
    \newtheorem*{claim}{Claim}
    \newtheorem*{example}{Example}
    \newtheorem*{examples}{Examples}
    \newtheorem*{exercise}{Exercise}
}{ 
    \newtheorem{theorem}{Théorème}[section]
    \newtheorem{definition}[theorem]{Définition}
    \newtheorem{notation}[theorem]{Notation}
    \newtheorem{corollary}[theorem]{Corrolaire}
    \newtheorem{proposition}[theorem]{Proposition}
    \newtheorem{lemma}[theorem]{Lemme}
    \newtheorem*{remark}{Remarque}
    \newtheorem*{claim}{Assertion}
    \newtheorem*{example}{Exemple}
    \newtheorem*{examples}{Exemples}
    \newtheorem*{exercise}{Exercise}
}


\surroundwithmdframed[style=theoremstyle,backgroundcolor=myRed]{theorem}
\surroundwithmdframed[style=theoremstyle,backgroundcolor=myGreen]{definition}

%%%%%%%%%%%%%%%%%%%
%   Parenthesis   %
%%%%%%%%%%%%%%%%%%%

\newcommand{\parenthesis}[1]{\left(#1\right)}
\newcommand{\bigparenthesis}[1]{\big(#1\big)}
\newcommand{\Bigparenthesis}[1]{\Big(#1\Big)}

\newcommand{\brackets}[1]{\left[#1\right]}

%%%%%%%%%%%%%%%%%%%
%    Operators    %
%%%%%%%%%%%%%%%%%%%

\DeclareMathOperator{\id}{id}
\DeclareMathOperator{\re}{Re}
\DeclareMathOperator{\Aut}{Aut}
\DeclareMathOperator{\Hom}{Hom}
\DeclareMathOperator{\Char}{char}
\DeclareMathOperator{\im}{Im}
\DeclareMathOperator{\dom}{dom}
\DeclareMathOperator{\cof}{cof}
\DeclareMathOperator{\Ob}{Ob}
\DeclareMathOperator{\Mor}{Mor}
\DeclareMathOperator{\rk}{rk}

\newcommand{\concat}{
  \mathord{
    \mathchoice
    {\raisebox{1ex}{\scalebox{.7}{$\frown$}}}
    {\raisebox{1ex}{\scalebox{.7}{$\frown$}}}
    {\raisebox{.7ex}{\scalebox{.5}{$\frown$}}}
    {\raisebox{.7ex}{\scalebox{.5}{$\frown$}}}
  }
}
\newcommand{\restr}[1]{\mathbin\upharpoonright_{#1}}
\newcommand{\compl}[1]{#1^\mathsf{c}}
\newcommand{\abs}[1]{\left| #1 \right|}
\newcommand{\bigabs}[1]{\big\lvert#1\big\rvert}
\newcommand{\parens}[1]{\left(#1\right)}
\newcommand{\norm}[1]{\left\| #1 \right\|}

%%%%%%%%%%%%%%%%%%%
%  Constructions  %
%%%%%%%%%%%%%%%%%%%

% Arrows
\newcommand{\xto}[1]{\xrightarrow{#1}}

% Structures 
\newcommand{\verteqsign}{\rotatebox{90}{$\,=$}}
\newcommand{\verteq}[2]{\underset{\scriptstyle\overset{\mkern4mu\verteqsign}{#2}}{#1}}
\newcommand{\vertimplies}[2]{\underset{\scriptstyle\overset{\uparrow}{#2}}{#1}}
\newcommand{\func}[5]{ 
    \begin{array}{rccl}
            #1 : & #2 & \longrightarrow & #3 \\
                 & #4 & \longmapsto     & #5
    \end{array} 
}
\newcommand{\ntimes}[2]{\underbrace{#1, \dots, #1}_{#2 \ttimes}}

% Set definitions
\newcommand{\set}[1]{\left\{ #1 \right\}}
\newcommand{\bigset}[1]{\big\{ #1 \big\}}
\newcommand{\setst}[2]{\left\{ #1 \, \middle| \, #2 \right\}}
\newcommand{\bigsetst}[2]{\big\{ #1 \, \big| \, #2 \big\}}

\newcommand{\funcset}[2]{{}^{#1}{#2}}
\newcommand{\angles}[1]{\left\langle #1 \right\rangle}
\newcommand{\brz}{B_r(z_0)}
\newcommand\quotient[2]{
    \mathchoice
        {% \displaystyle
            \text{\raise1ex\hbox{$#1$}\Big/\lower1ex\hbox{$#2$}}%
        }
        {% \textstyle
            #1\,/\,#2
        }
        {% \scriptstyle
            #1\,/\,#2
        }
        {% \scriptscriptstyle  
            #1\,/\,#2
        }
}
\newcommand{\interior}[1]{\mathring{#1}}

%%%%%%%%%%%%%%%
%   Letters   %
%%%%%%%%%%%%%%%


% Greek alphabe
\renewcommand{\phi}{\ensuremath\varphi\xspace}
\renewcommand{\epsilon}{\ensuremath\varepsilon\xspace}
\renewcommand{\a}{\ensuremath\alpha\xspace}
\renewcommand{\d}{\ensuremath{\partial}\xspace}
\newcommand{\D}{\ensuremath\Delta\xspace}
\newcommand{\e}{\ensuremath\epsilon\xspace}
\newcommand{\g}{\ensuremath\gamma\xspace}
\newcommand{\y}{\g}  % I mistake them every other time
\newcommand{\p}{\phi\xspace}
\newcommand{\s}{\sigma\xspace}
\newcommand{\w}{\ensuremath\omega\xspace}

% Mathbb alphabet
\newcommand{\C}{\ensuremath{\mathbb C}\xspace}
\newcommand{\F}{\ensuremath{\mathbb F}\xspace}
\newcommand{\IG}{\ensuremath{\mathbb G}\xspace}
\newcommand{\N}{\ensuremath{\mathbb N}\xspace}
\newcommand{\R}{\ensuremath{\mathbb R}\xspace}
\newcommand{\IP}{\ensuremath{\mathbb P}\xspace}
\newcommand{\Q}{\ensuremath{\mathbb Q}\xspace}
\newcommand{\Z}{\ensuremath{\mathbb Z}\xspace}

% Fancy fonts
\newcommand{\Aa}{\ensuremath{\mathcal A}\xspace}
\newcommand{\Bb}{\ensuremath{\mathcal B}\xspace}
\newcommand{\Cc}{\ensuremath{\mathcal C}\xspace}
\newcommand{\calG}{\mathcal{G}\xspace}
\newcommand{\calK}{\mathcal{K}\xspace}
\newcommand{\Dd}{\ensuremath{\mathcal D}\xspace}
\newcommand{\Ff}{\mathcal{F}}
\newcommand{\Pp}{\mathcal{P}}
\newcommand{\Mm}{\mathcal{M}}
\newcommand{\Nn}{\mathcal{N}}
\newcommand{\Tt}{\ensuremath{\mathcal T}\xspace}
\newcommand{\Ss}{\ensuremath{\mathcal S}\xspace}
\newcommand{\Uu}{\ensuremath{\mathcal U}\xspace}
\renewcommand{\O}[1]{\mathcal{O}\left(#1\right)}

% Others
\renewcommand{\emptyset}{\varnothing}


%%%%%%%%%%%%%%%%%%
%   New symbols  %
%%%%%%%%%%%%%%%%%%

\newcommand{\iddots}{{\cdot^{\cdot^\cdot}}}  % Three dots on the antidiagonal
\newcommand{\existsinf}{\exists^\infty}

%%%%%%%%%%%%%%%%%%
%   Shorthands   %
%%%%%%%%%%%%%%%%%%

% Math text
\newcommand{\open}{\text{ open }}
\newcommand{\st}{\text{ \ifenglish{s.t}{t.q}. }}
\newcommand{\andd}{\text{ \ifenglish{and}{et} }}
\newcommand{\andquad}{\quad\andd\quad}
\newcommand{\et}{\text{ et }}
\newcommand{\ttimes}{\text{ times}}
\newcommand{\tif}{\text{if }}
\newcommand{\otherwise}{\text{otherwise}}

% Symbols
\newcommand{\acts}{\circlearrowright}
\newcommand{\sub}{\subset}
\newcommand{\subeq}{\subseteq}
\renewcommand{\iff}{\ensuremath{\Longleftrightarrow}}
\newcommand{\iffquad}{\quad\iff\quad}
\newcommand{\del}{\partial}
\newcommand{\x}{\times}
\renewcommand{\emph}{\textbf}

%%%%%%%%%%%%%%%%%%%%
% Subject specific %
%%%%%%%%%%%%%%%%%%%%

% Maybe those should go inside their respesective files

% Graph theory
\DeclareMathOperator*{\ex}{ex}
\newcommand{\Turan}{\mathcal{T}}

% Analytic number theory
\newcommand{\sumdn}{\sum_{d|n}}
\newcommand{\Li}{\mathrm{Li}}

% Smooth manifolds
\newcommand{\chart}[1]{(\phi_{#1}, U_{#1})}
\newcommand{\der}[1]{\frac{\del}{\del #1}}

% Galois Theory
\newcommand{\clot}[1]{\overline{#1}}
\newcommand{\Gal}{\mathrm{Gal}}
% \DeclareMathOperator*{\Gal}{Gal}
\newcommand{\Kxx}{K[x_1, \dots, x_n]}

% Algebraic topology
\newcommand{\HCW}{H^{\mathrm{CW}}}

% Complexity
\newcommand{\coP}{\mathsf{coP}}
\newcommand{\Ptime}{\mathsf{P}}
\newcommand{\NP}{\mathsf{NP}}
\newcommand{\coNP}{\mathsf{coNP}}
\newcommand{\PSPACE}{\mathsf{PSPACE}}
\newcommand{\NPSPACE}{\mathsf{NPSPACE}}


% Descriptive Set Theory
\newcommand{\baire}{\ensuremath{\w^\w}\xspace}
\newcommand{\cantor}{\ensuremath{2^\w}\xspace}
\newcommand{\W}{\ensuremath{\w_1}\xspace}  % This is meant to be used only for the first uncountable ordinal
% Classes
\newcommand{\bS}[1]{\ensuremath{\mathbf{\Sigma}^0_{#1}}\xspace}
\newcommand{\bP}[1]{\ensuremath{\mathbf{\Pi}^0_{#1}}\xspace}
\newcommand{\bD}[1]{\ensuremath{\mathbf{\Delta}^0_{#1}}\xspace}
\newcommand{\borelsigma}[1]{\ensuremath{\mathbf{\Sigma}^0_{#1}}\xspace}
\newcommand{\borelpi}[1]{\ensuremath{\mathbf{\Pi}^0_{#1}}\xspace}
\newcommand{\boreldelta}[1]{\ensuremath{\mathbf{\Delta}^0_{#1}}\xspace}
\newcommand{\borelGamma}{\ensuremath{\mathbf{\Gamma}}\xspace}
\newcommand{\LambdaClass}[2]{\ensuremath{\mathbf{\Lambda}_{#1, #2}}\xspace}
\newcommand{\dualclass}[1]{\check{#1}}
\newcommand{\nsd}{non-self-dual\xspace}
\newcommand{\projectivesigma}[1]{\ensuremath{\mathbf{\Sigma}^1_{#1}}\xspace}
\newcommand{\projectivepi}[1]{\ensuremath{\mathbf{\Pi}^1_{#1}}\xspace}
\newcommand{\projectivedelta}[1]{\ensuremath{\mathbf{\Delta}^1_{#1}}\xspace}
% Sets of sequences
\newcommand{\seqlt}[2]{{#2}^{<#1}}
\newcommand{\finiteseq}[1]{\seqlt{\w}{#1}}
% \newcommand{\infseq}[1]{\funcset{\omega}{#1}}
% \newcommand{\allseq}[1]{\funcset{\leq\omega}{#1}}
\newcommand{\infseq}[1]{{#1}^\w}
\newcommand{\allseq}[1]{{#1}^{\leq\omega}}
\newcommand{\finitebaire}{\ensuremath{\finiteseq{\w}}\xspace}
\newcommand{\finitecantor}{\ensuremath{\finiteseq{2}}\xspace}
\newcommand{\length}[1]{\mathrm{lh}(#1)}
\newcommand{\emptyseq}{\langle\rangle}
% Operators
\renewcommand{\succ}[2][]{\ifthenelse{\isempty{#1}}{\mathrm{succ}(#2)}{\mathrm{succ}_{#1}(#2)}}
\DeclareMathOperator{\odd}{odd}
\DeclareMathOperator{\even}{even}
\DeclareMathOperator{\last}{last}
\newcommand{\Countf}[1]{\#_{#1}}
\newcommand{\difference}[1]{D_{#1}}
\newcommand{\suffixes}[2]{{#1}_{(#2)}}
\newcommand{\wadd}{\oplus}
\newcommand{\wCountableSup}{\mathrm{sup^\w}}
\newcommand{\wmult}{\mathop{\otimes}}
\newcommand{\bigeraser}{\ensuremath{\mathord{\Uparrow}}\xspace}
\newcommand{\interleave}{\mathbin{\S}}
\newcommand{\Init}[1]{\mathrm{Init}_{#1}}
% Players
\newcommand{\I}{\texttt{I}\xspace}
\newcommand{\II}{\texttt{II}\xspace}
\newcommand{\playerA}{Player \I}
\newcommand{\playerB}{Player \II}

% \renewcommand{\I}{\texttt{M}\xspace}
% \renewcommand{\II}{\texttt{B}\xspace}
% \renewcommand{\playerA}{Michael\xspace}
% \renewcommand{\playerB}{Bob\xspace}
% Wadge
\newcommand{\leW}{\le_\textsc{w}}
\newcommand{\geW}{\ge_\textsc{w}}
\newcommand{\gtW}{>_\textsc{w}}
\newcommand{\ltW}{<_\textsc{w}}
\newcommand{\equivW}{\equiv_\textsc{w}}
\newcommand{\Wdegree}[1]{[#1]_{\equivW}}
% Axioms / Theories
\newcommand{\ZF}{\textbf{ZF}\xspace}
\newcommand{\ZFC}{\textbf{ZFC}\xspace}
\newcommand{\AD}{\textbf{AD}\xspace}
\renewcommand{\AC}{\textbf{AC}\xspace}
\newcommand{\DC}{\textbf{DC}\xspace}
% Games
\newcommand{\GaleStewart}[1]{\ensuremath{G(#1)}\xspace}
\newcommand{\BanachMazur}[1]{\ensuremath{BM(#1)}\xspace}
\newcommand{\Wadge}[2]{\ensuremath{G(#1, #2)}\xspace}
% Automatons
\newcommand{\automaton}{automatic set\xspace}
\newcommand{\automata}{automatic sets\xspace}
\newcommand{\Automata}{Automatic sets\xspace}
\newcommand{\Language}{\mathcal{L}}
\DeclareMathOperator{\Inf}{Inf}
\newcommand{\AcceptanceCondition}{\mathrm{Acc}}
\newcommand{\automatonMinus}{\scalebox{0.5}{\begin{tikzpicture}[automata, baseline=-2mm]
    \node[minus] {};
\end{tikzpicture}}\xspace}
\newcommand{\automatonPlus}{\scalebox{0.5}{\begin{tikzpicture}[automata, baseline=-2mm]
    \node[plus] {};
\end{tikzpicture}}\xspace}
\newcommand{\automatonState}[1]{\scalebox{0.5}{\begin{tikzpicture}[automata, baseline=-2mm, minimum size=20]
    \node[state] {\Large \(#1\)};
\end{tikzpicture}}\xspace}
\newcommand{\sumAutomaton}[2]{\begin{tikzpicture}[automata]
    \path node[initial, state] (root) {$#2$} 
        +(2, 1) node[state] (A) {$#1$}
        +(2, -1) node[state] (Ac) {$\compl{#1}$};
    \draw (root) edge (A) edge (Ac);
\end{tikzpicture}}


%%%%%%%%%%%%%%%%%
%     Tikz      %
%%%%%%%%%%%%%%%%%

\tikzcdset{row sep/normal=1cm, column sep/normal=1cm}  % So tikz-cd diagram are more squared

% Draws irregular circles.
\newcommand{\irregularcircle}[2]{  % radius, irregularity
  let \n1 = {(#1)+rand*(#2)} in
  +(0:\n1)
  \foreach \a in {10,20,...,350}{
    let \n1 = {(#1)+rand*(#2)} in
    -- +(\a:\n1)
  } -- cycle
}

\tikzset{
triangle/.style={ % To create triangles in trees
  draw=myRed!90!black,
  text=black,
  fill=myRed,
  shape border uses incircle,
  isosceles triangle,
  shape border rotate=90,
}}

% Used to make arrows on paths. Use like this: \draw[marr=\Singlearrow]
\tikzset{marr/.style={
  decoration={
    markings,
    mark=at position 0.5 with {#1}
    },
  postaction={decorate}
}}
\def\Singlearrow{{\arrow[scale=1.5,xshift={0.5pt+2.25\pgflinewidth}]{>}}}
\def\Doublearrow{{\arrow[scale=1.5,xshift=1.35pt +2.47\pgflinewidth]{>>}}}
\def\Triplearrow{{\arrow[scale=1.5,xshift=1.75pt +2.47\pgflinewidth]{>>>}}}
\newcommand{\CWarrow}[3]{ [->] ($1/3*(#1) + 1/3*(#2) + 1/3*(#3)$) ++(220:4pt) arc (220:-40:4pt)}
\newcommand{\CCWarrow}[3]{ [->] ($1/3*(#1) + 1/3*(#2) + 1/3*(#3)$) ++(-40:4pt) arc (-40:220:4pt)}

% For automatons
\tikzset{
    initial text={},
    do path picture/.style={%
        path picture={%
        \pgfpointdiff{\pgfpointanchor{path picture bounding box}{south west}}%
            {\pgfpointanchor{path picture bounding box}{north east}}%
        \pgfgetlastxy\x\y%
        \tikzset{x=\x/2,y=\y/2}%
        #1
        }
    },
    automata/.style={
        node distance=60,  % I don't know what the unit is. I would like 2.5cm, but that can scale
        minimum size=20,
        every edge/.style={
            draw,
            ->,
            auto,
        }
    },
    plus/.style={
        draw,
        circle,
        do path picture={    
            \draw [line cap=round, very thick] (-1/2,0) -- (1/2,0) (0,-1/2) -- (0,1/2);
        }
    },
    minus/.style={
        draw,
        circle,
        do path picture={    
            \draw [line cap=round, very thick] (-1/2,0) -- (1/2,0);
        }
    },
    % Removes extra space on initial nodes
    % From: https://tex.stackexchange.com/questions/111554/how-to-avoid-superfluous-space-in-the-initial-state-of-a-tikz-automaton#comment245757_111558
    every initial by arrow/.append style={anchor/.append style={shape=coordinate}},
}

% To put at the bottom right corner of a square that commutes.
\newcommand{\commutes}{\arrow[ul, phantom, "\scalebox{1.5}{$\circlearrowleft$}"]}

% To link two lines in a long exact sequence.
% First argument is where the line should connect (ex: dll)
% Second argument is the label on the line.
\newcommand{\connecting}[2]{
    \arrow[#1,phantom, ""{coordinate, name=Z}]
    \arrow[#1, "#2", rounded corners,
        to path={
            -- ([xshift=2ex]\tikztostart.east)
            |- (Z)[near start]\tikztonodes
            -| ([xshift=-2ex]\tikztotarget.west)
            -- (\tikztotarget)
        }]
}
\newcommand{\dotsto}{\mathllap{\cdots\longrightarrow\ }}  % ... → indicate the start of a long sequence without taking space
\newcommand{\todots}{\mathrlap{\ \longrightarrow\cdots}}  % → ... indicate the end of a long sequence without taking space 

\newenvironment{longsequence}{
    \begin{center}
        \begin{tikzcd}[row sep=0.5cm]
}{
        \end{tikzcd}
    \end{center}
}


\newcommand{\quickfig}[2]{
    \begin{wrapfigure}{r}{40mm}
        \begin{center}
            #2
        \end{center}
        \caption{#1}
    \end{wrapfigure}
}



\title{Homework 1}
\author{Diego Dorn}
% \date{Automn 2021}

\begin{document}
    \maketitle

    % \tableofcontents

    \section*{Exercise 1}

    For a decisive Turing machine $\Mm$, we write $L(\Mm)$ for the language it decides.
    We first show that if $L(\Mm) \in \NP$, then $L(\Mm) \in \coNP$.

    \begin{claim}
        If 
        $L(\Mm) \in \NP$ 
        for all decisive Turing machine $\Mm$, 
        then for all decisive Turing machine $\Nn$,
        $L(\Nn) \in \coNP$.
    \end{claim}

    \begin{proof}
        Let $\Mm$ be a decisive Turing machine,
        we build a decisive Turing machine $\compl{\Mm}$
        that runs exactly the same way as $\Mm$, but on each branch,
        when it halts it outputs \textit{yes} when $\Mm$ outputs \textit{no}
        and \textit{no} when $\Mm$ outputs \textit{yes}. The \textit{maybe} 
        do not change. This way $\compl{\Mm}$ accepts a word $x$ if and only
        if $\Mm$ does not accept $x$. 
        Thus $L(\compl{\Mm}) = \compl{L(\Mm)}$.
        Using the hypothesis with $\compl{\Mm}$, 
        we get that $L(\compl{\Mm}) = \compl{L(\Mm)} \in \NP$
        and therefore $L(\Mm) \in \coNP$.
    \end{proof}

    We now prove that for a polytime decisive Turing machine $\Mm$, 
    $L(\Mm) \in \NP$.
    To that extent, we construct a polytime verifier $V(x, C)$ for $L$
    as follows:
    \begin{itemize}
        \item If $C$ doesn't encode a sequence of transitions of $\Mm$, reject $x$. 
            Otherwise, let $(C_n)$ be the finite sequence of transitions encoded by $C$.
        \item Simulate $\Mm$ on input $x$ and on the $n$-th step, use $C_n$ to decide 
            of the next state of the deterministic simulation. 
            If the transition $C_n$ is not in the transition function of $\Mm$, reject $x$.
        \item When the simulation of $\Mm$ halts, if it outputs $yes$, 
            accept $x$. Otherwise, reject $x$.
    \end{itemize}

    Since we can simulate Turing machines with only a polynomial slowdown,
    and $\Mm$ is a polytime NTM, 
    it is clear that $V$ run in polynomial time.
    Thus, we only need to prove that $\Mm$ accepts $x$ if and only if there exist $C$
    such that $V(x, C)$ accepts $x$. 

    First, we notice that on any given input $x$, there cannot be one branch 
    of the computation that outputs \textit{yes} and an other one that outputs \textit{no}.
    
    For the first direction, there is a branch of the computation of 
    $\Mm$ on input $x$ that halts on a \textit{yes} or $\textit{no}$,
    so it suffices to take $C$ as the sequence of transitions
    that make up that branch.

    For the reverse direction, if there is a $C$ that make $V$ output \textit{yes},
    it must be a valid sequence of transitions for $\Mm$, 
    and when $\Mm$ follows this branch, it outputs \textit{yes}.
    Since there is a branch of $\Mm$ that outputs \textit{yes},
    all the other branches must outputs \textit{yes} or \textit{maybe}
    and thus $\Mm$ accepts $x$.


    \section*{Exercice 2}

    First we show that \textsc{SepMatch} is in $\NP$.
    We can define a verifier $V(G, k, C)$ 
    for \textsc{SepMatch} with the certificate $C$ being
    an encoding of the edges of the matching.
    $V$ checks for all pair of edges if the distance
    between each pair of points is less than 2 in $G$, 
    and that there are exactly $k$ edges. This can be done
    in $\O{|G|^3}$.

    To show that \textsc{SepMatch} is $\NP$-hard,
    we reduce \textsc{IndSet} to it.

    Let $G = (V, E)$ be any finite graph. We define the graph $G' = (V', E')$
    as \begin{itemize}
        \item $V' = V \times \set{0, 1}$
        \item $E' = \setst{
            \bigparenthesis{(u, i), (v, j)} \in V' \times V'
        }{
            (u, v) \in V \vee (u = v \wedge j = 1 - i)
        }$, that is, two vertices are connected in $G'$
        if their first coordinate are connected in $G$
        or if they their first coordinate is the same and the second coordinate
        are different.
    \end{itemize}

    Our reduction from \textsc{IndSet} to \textsc{SepMatch}
    if the function that maps and input of \textsc{IndSet}
    $\angles{G, k}$ to the input of \textsc{SepMatch} $\angles{G', k}$.

    Before proving that this is indeed a reduction, we notice that this map 
    is computable in polynomial time and we setup some notations and a lemma.
    Let $d_G$ and $d_{G'}$ the distances between vertices of $G$ and $G'$
    respectively. For a node $u \in G'$, we write $u = (u_0, u_1)$,
    and $\pi_0$ the projection on the first coordinate, so $\pi_0(u) = u_0$.

    \begin{lemma*}
        For all $u, v \in V'$ with $u_0 \neq v_0$, \[
            d_{G'}(u, v) = d_G(u_0, v_0).
        \]
    \end{lemma*}
    \begin{proof}
        Let $u, v \in V'$. We first show that $d_{G'}(u, v) \geq d_G(u_0, v_0)$.
        Indeed, let $\w = \w_0\w_1\dots\w_n$ be a path from $u$ to $v$ in $G'$. 
        We have that $\w_G = \pi_0(\w_0) \pi_0(\w_1) \dots \pi_0(\w_n)$
        is a path in $G$ if we remove vertices that may appear more than once consecutively.
        Thus $\abs{\w_G} \leq \abs{\w}$ and the distances in $G$ are always 
        less than the distances in $G'$.

        It remains to be shown that $d_{G'}(u, v) \leq d_G(u_0, v_0)$.
        Let $\w = w_0\w_1\dots\w_n$  be a path from $u_0$ to $v_0$ in $G$. We
        construct a path $\w_{G'}$ from $u$ to $v$ of the same length 
        in $G'$. For $i = 0, \dots, n$,
        \[ 
            (\w_{G'})_i = \begin{cases}
                (\w_0, u_1) & \tif i = 0 \\
                (\w_i, v_1) & \tif i > 0
            \end{cases}
        \]
        This is clearly a path in $G'$ from $(\w_0, u_1) = u$ to $(\w_n, v_1) = v$
        since at any step $\w_i$ is a neighbour of $\w_{i+1}$.
        Thus distances in $G'$ are at most distances in $G$.
    \end{proof}

    We now show that $\angles{G, k} \in \textsc{IndSet}$ if and only if 
    $\angles{G', k} \in \textsc{SepMatch}$.
    
    \paragraph{$\implies$}
    Let $S \subset V$ be an independent set of size $k$,
    then $M = \setst{
        \bigparenthesis{
            (s, 0), (s, 1)
        }}{
            s \in S
        }
    $ is a separated matching (of size $k$) in $G'$.
    Indeed, let $e, e' \in M$ be two different edges.
    We know that for all $u \in e$ and $v \in e'$, 
    \begin{itemize}
        \item $u_0 \neq v_0$ since the edges are different.
        \item $d_G(u_0, v_0) \geq 2$ since $S$ is an independent set.
    \end{itemize}
    So by the lemma, $d_{G'}(u, v) = d_G(u_0, v_0) \geq 2$
    and $M$ is indeed a separated matching.

    \paragraph{$\impliedby$}
    If $\angles{G', k} \in \textsc{SepMatch}$, 
    let $M$ be a separated matching of size $k$. 
    We show that 
    \[
        S = \setst{u_0}{(u, v) \in M}
    \]  
    is an independent set (of size $k$) in $G$.
    To that extent, let $u_0, u_0' \in S$
    and $(u, v), (u', v') \in M$ the corresponding edges.
    Since $M$ is a separated matching, 
    we have that $d_{G'}(u, u') \geq 2$
    so that, by the lemma, $d_G(u_0, u_0') = d_{G'}(u, u') \geq 2$,
    so $u_0$ and $u_0'$ are indeed separated.


    \section*{Exercice 3}

    \paragraph{1.}
    Let \Uu be a universal Turing machine that can
    simulate an other Turing machine with a polynomial slowdown.
    We set $A$ as the same oracle as in class, that is, 
    \[
        A = \setst{ \angles{M, 1^k} }{
            \Uu\parenthesis{
                \angles{M, \angles{M, 1^k}}
            } = 0
            \text{ in } 2^k \text{ steps}
        }.
    \]
    We have seen in class that $\Ptime^A = \NP^A$, so we just show that
    $\coNP^A = \NP^A$. Indeed, 
    \begin{align*}
        L \in \coNP^A 
        &\iff 
        \compl{L} \in \NP^A
        \\& \iff 
        \compl{L} \in \Ptime^A
        \\& \iff 
        L \in \coP^A = \NP^A.
    \end{align*}
    Were the last line follows from the fact that if a language can be 
    decided deterministically in polynomial time, then its complement
    can be too, by just inverting the outputs of the Turing machine.

    \paragraph{2.}
    Let $ \Bb = \left\{ B : \forall n \abs{B \cap \set{0, 1}^n} = 1 \right\} $,
    and for $B \in \Bb$, let $
        L_B = \setst{1^n}{ \exists x \in \set{0, 1}^{n-1}, 0x \in B}
    $.

    \begin{claim}
        For all $B \in \Bb$, $L_B \in \NP$.
    \end{claim}
    \begin{proof}
        If $1^n \in L_B$, a certificate is the $x \in \set{0, 1}^{n-1}$
        such that $0x \in B$. Indeed, a verifier can check if $0x \in B$ in a 
        single query to the oracle.
    \end{proof}


    \begin{claim}
        For all $B \in \Bb$, $L_B \in \coNP$.
    \end{claim}
    \begin{proof}
        We have that \begin{align*}
            \compl{L_B} &=
            \setst{1^n}{ 
                \forall x \in \set{0, 1}^{n-1}, 0x \notin B
            }
            \\&=
            \setst{1^n}{ 
                \exists x \in \set{0, 1}^{n-1}, 1x \in B
            }
        \end{align*}
        where the last equality follows from the the fact that
        $\abs{B \cap \set{0, 1}^n} = 1$.
        With $\compl{L_B}$ written as such, we can apply the same proof as the previous
        claim, mutatis mutandis.
    \end{proof}

    \begin{claim}
        There exist $B \in \Bb$ such that $L_B \notin \Ptime^B$.
    \end{claim}
    \begin{proof}
        Fix an enumeration $(M_n)_{n \in \N}$ of all oracle compatible Turing machines,
        and let $b: \N \to \N \times \N$ be any bijection.
        We define the sequence $\Mm_n = M_{b(n)_0}$ that contains each Turing machine
        infinitely many times.

        We now construct $B$ so that it fools each machine $M_n$ that runs in
        polynomial time.
        Fix $n \in N$, we now construct the $x \in \set{0, 1}^n$
        that belongs to $B$.
        We run $\Mm_n$ on $1^n$, and let $Q_k$ be the set of queries 
        that $\Mm_n$ has asked up to the $k$-th
        step of execution (not included).
        For $i \in \set{0,1}$, we define $Q_k^i = \setst{q \in Q_k^i}{\abs{q} = n \andd q_0 = i}$
        the set of queries of length $n$ that start with a $i$.
        At step $k$, if the machine queries ``$q \in B$?'':
        \begin{itemize}
            \item if $q \notin \set{0, 1}^n$, we can \textit{yes} if and only 
                if $q = 0^{\abs{q}}$, which would not change the output of 
                a $\Mm_n$ on input $x$ that recognises any fixed $B$.
            \item if $q \in Q_k$, the query has been already made, so 
                we answer the same as the previous time the machine made this query.
            \item if $q \notin Q_k$ and for some $i \in \set{0, 1}$, $\abs{Q_k^i} = 2^n$, 
                that is, the set $i\concat \set{0, 1}^{n-1}$
                was completely queried before, we must have $q \in (1-i)\concat \set{0, 1}^{n-1}$
                so we say say that $q \notin B$.
            \item otherwise, if $\abs{Q_k^{q_0}} = 2^n -1$, that is, 
                $q$ is the last non-queried entry that start with $q_0$,
                we say that $q \in B$
            \item otherwise we say that $q \notin B$.
        \end{itemize}

        When the machine halts, if it hasn't queried all of both
        $i \concat \set{0, 1}^{n}$, for $i = 0, 1$,
        if the machine outputs: \begin{itemize}
            \item \textit{yes},
                let $x \in \setst{1y \in \set{0, 1}^n}{\forall k, 1y \notin Q_k^1}$.
            \item \textit{no},
                let $x \in \setst{0y \in \set{0, 1}^n}{\forall k, 0y \notin Q_k^0}$.
        \end{itemize}
        We set $B \cap \set{0, 1}^{n} = \set{x}$, which is a number that $\Mm_n$
        did not query and that make its prediction fail.

        Thus, with the $B$ as constructed, if a machine $\Mm_n$ doesn't query 
        at least all the sequences in $\set{0, 1}^n$ that start with 0
        or all the sequences that start with 1,
        it fails.
        Therefore we only need to show that every polytime machine $M_N$
        don't query enough the oracle on large inputs.
        Indeed, fix a polytime machine $M_N$
        and define a subsequence
        $(\Mm_{n_k})_{k \in \N} \subset (\Mm_n)_{n \in \N}$ 
        such that for all $k \in \N$, $M_{n_k} = M_N$.
        However, since $n^{\O1} << \O{2^n}$, there must be some $k \in \N$,
        such that $\Mm_{n_k}$ doesn't make $2^{n_k} - 1$ queries and therefore fails.
        Thus, $M_N$ fails on input $1^{n_k}$, and there is no polynomial time
        machine that decides $L_B$.
    \end{proof}

    Therefore, $L_B \in \NP^B \cap \coNP^B \setminus \Ptime^B$.


    \newpage
    .
\end{document}
